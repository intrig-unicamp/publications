\section{Conclusions}
\label{sec:Conclusion}

The traditional telecommunication industry is facing multiple challenges to keep competitive and improve the mode network services are designed, deployed and managed. Architectures and enabling technologies such as Cloud Computing, SDN and NFV, are providing new paths to overcome these challenges in a software-driven approach.  \acrfull{nso} is a strategic element to converge various technology domains and provide a broader and more agile network service footprints. 

In this comprehensive survey on network service orchestration, we highlight its growing importance and try to contribute to an overarching understanding of the common concepts and diverse approaches towards practical embodiments of NSO. We present enabling technologies, clarify the definitions behind the term orchestration, review standardization advances, research projects, commercial solutions, and list a number of open issues and research challenges. 

The application of NSO in some scenarios was also presented, where it is possible to sense its potential and understand the motivation behind so much ongoing work. We also observe a growing trend towards the use of open source components or solutions in orchestration platforms; however, the platforms require to evolve until become suitable for production. An important contribution of this work was the definition of a taxonomy that categorizes the leading characteristics and features related to network service orchestration.

Despite the fast pace of this vibrant topic, we expect this survey to serve as a guideline to researchers and practitioners looking into an overview of network service orchestration fundamentals, a reference to relevant related work and pointers to open research questions.