\section{Application Scenarios}
\label{sec:scneario}

\gls{nso} is envisaged to support diverse use case scenarios. This section aims at providing a brief practical view on a number of application domains and the main benefits provided by \gls{nso} in each scenario, delivering a sample of the expected potential of NSO in operation. 

\subsection{Next Generation Mobile Telecommunication Networks}

The fifth generation of mobile communication systems (5G) is expected to meet diverse and stringent requirements that are currently not supported by current mobile telecommunication networks, like ubiquitous connectivity (connectivity available anywhere), zero latency (lower than few milliseconds) and high-speed connection (10 times higher than 4G).

An efficient realization of 5G requires a flexible and programmable infrastructure covering transport, radio, and cloud resources~\cite{NGMN:5G:2017}. \gls{sdn} and \gls{nfv} are considered key enabling technologies to provide the required flexibility in processing and programmability, whereas end-to-end orchestration is regarded fundamental to improve the mobile service creation and resource utilization across all network segments, from radio access to transport~\cite{rostami-ran-transport-17}.

Furthermore, end-to-end orchestration should tackle  a significant challenge in mobile telecommunication networks, namely, the integration of different technologies, including radio, \gls{sdn} and \gls{nfv} so that network services may be dynamically created and adapted across the domains (wireless, aggregation and core). 

Finally, mobile management and orchestration solutions are expected to enable (i) congestion handling per subscriber or traffic, (ii) dynamic allocation of resources according to traffic variation and/or service requirements, and (iii) load reduction on transport networks and central processing units~\cite{EricssonInc.2015}. 

Future mobile/5G and fixed networks scenarios with diverse service requirements represent a growing and more complex challenge at the time of managing network resources. Network Slicing is being widely discussed in standard organizations as an essential mechanism to provide flexibility in the management of network resources~\cite{NGMN:5G:2017}. Network Slicing enables operators to create multiple network resources and (virtual) network functions isolated and customized over the same physical infrastructure~\cite{Galis:2018}. Such dedicated networks, built on a shared infrastructure can reduce the cost of the network deployment, speeds up the time to market and offer individual networks customizations according to customer requirements so that operators can introduce new market services~\cite{Devlic2017NESMO:Framework}. 

Increased flexibility introduces higher complexity in design and operation of network slices. Keys to avoid the CAPEX and OPEX increase is to automate the full lifecycle phases of a slice: (i) preparation phase, (ii) instantiation, configuration and activation phase, (iii) runtime phase and (iv) decommissioning phase~\cite{3GPP:TR28801:2017}. Besides the automation, other management and orchestration use cases of network slicing are fault management, performance management, and policy management. It is also expected multi-operator coordination management in order to create end-to-end network slices across multiple administrative domains and some level of management to be exposed to the network slice tenant~\cite{Contreras:2018}.

\subsection{Transport Networks}
Optical networks evolved from statically assigned single and multi-mode fiber channels to highly flexible modulation schemes using separate wavelengths. Nowadays, the optical equipment allows prompt wavelength conversion and flexible packet-to-optical setups. Given that agility increase, more programmability is being added to optical networks, for instance through PCE-based architectures for application-based network operations (ABNO)~\cite{RFC7491}. 

Under the flag of Software-Defined Optical Networks~\cite{7503119}, such as those based on OpenFlow extensions, different use cases target transport networks to deliver new approaches on wavelength-based routing and virtualization of optical paths.  Like \gls{pce}, different forms of \gls{sdn} abstractions in optical networks come with a logically centralized entity to program network elements encompassing optical paths. In a wider perspective, logical services are implemented through central controllers as part of a \gls{nso} workflow.
 
Optical transport of traffic across long-range areas, from data centers to end customers as Fiber-to-the-X (e.g., houses FTTH, curbs FTTC, Nodes FTTN), involve different intermediate elements requiring packet-optical conversions and vice-versa. An \gls{nso} envisioned in this scenario of packet-optical integration can take advantage of the knowledge about topology and equipment status, therefore optimizing traffic forwarding according to optical and packet-oriented capabilities. For instance, an \gls{nso} could optimize and aggregate \gls{mpls} \glspl{lsp} inside optical transport networks as part of higher-level service lifecycle goals.

Ongoing work at \gls{mef} aims to standardize \gls{sdwan} \cite{MEF:SDWAN:2017}  as the means to flexibly achieve programmable micro-segmented paths -- based on QoS, security and business policies -- across sites (public or private clouds), using overlay tunnels over varied underlay technologies, such as broadband Internet and MPLS. A service orchestrator is needed to tailor and scale paths on-demand to assure application policies by interfacing a controller that manages programmable edge \gls{sdwan} routers, spanning multiple provider sites. WAN traffic can flow through non-trusting administrative domains in heterogeneous wired/wireless underlay networks with different performance metrics.

\subsection{Cloud Data Centers}

Data Centers have long been upgraded with network virtualization for traffic forwarding and scaling L2 domains, such as VXLAN. Current technologies realize hypervisor tunneling for north--south and east--west traffic in data centers. More importantly, with the advent of operating system-level virtualization (a.k.a containers), even more flexible methods of end-host network virtualization have been deployed in data centers --- there are examples already available in commercial products (e.g., VMWare NSX). In addition, computer virtualization platforms also contain networking extensions/plugins for dynamic networking between servers (e.g., Kubernetes and OpenStack). Those logically programmable network fulfillments derive the properties that concern a \gls{nso}.

The orchestration of cloud resources~\cite{liu2011cloud} has been a longstanding topic of research and actual commercial solutions.  \gls{nso} programmability has been increasingly important to keep isolation in-network and at servers for heterogeneous customers that inhabit public clouds (e.g., Azure, AWS and Google Cloud). For instance, Kubernetes, using kube-proxy, defines networking in Google Cloud via a set of dynamic routes associations between service addresses and bridges' addresses in PODs (servers) hosting containers; ideally, a service is maintained independently of the associated containers host location. Container-based orchestration is a production reality, but many challenges remain open~\cite{7185168}, a number of them related to the seamless integration with network services inside the data centers and across data centers.

Similar concepts of \gls{nso} characteristics already exist to program paths optimizing traffic workloads, high throughput and low latency across data centers and to edge \glspl{cdn} -- best examples being Google B4 and Andromeda \gls{sdn} projects. Therefore, \gls{nso} already plays an essential role in data center networking as it became a pioneer in direct application of \gls{sdn} concepts. 

Lately, research topics in this domain concern integration of multiple cloud environments envisioning different guarantees of \gls{sla} for distinct classes of traffic. As more mobile applications evolve towards accomplishing customers requirements for low latency and high throughput (e.g., virtual and augmented reality), \gls{nso} will play an important role in addressing issues originated from those requirements.  

\subsection{Internet of Things}
According to Gubbi et al.~\cite{Gubbi2013InternetDirections}, \gls{iot} is a network of sensing and actuating devices providing the ability to share information through a unified platform. Such devices or "things" may transmit a significant amount of data over a network without requiring human-to-human or human-to-computer interaction. Its application areas include homes, cities, industry, energy systems, agriculture, and health. Due to the amount of generated-data and its dynamic and transient operational behavior, \gls{iot} will lead to scalability and management issues in the process of transport, processing, and storage of the data in real time~\cite{Mijumbi2016NetworkChallenges}. Besides, the various entities involved need to be orchestrated to convert the data into actionable information~\cite{Consel2017InternetOrchestration}. 

\gls{nso} along with \gls{nfv} and \gls{sdn} allow network services to be automatically deployed and managed. In this scenario, \gls{sdn} is responsible for establishing the network connections, \gls{nfv} provides the management of the network functions, and \gls{nso} govern all deployment process of the end-to-end network service. Such paradigms can help to process and manage significant amounts of IoT-generated data with better network efficiency. The separation between resources and services provided by such technologies enables the isolation and lower impact risks of \gls{iot} on other infrastructures. Also, they can reduce human intervention in the operation of the network, feature that is essential to the achievement of \acrlong{iot}.

The authors in~\cite{Wen2017FogServices} propose an orchestrator for \acrlong{iot} that manages all planes of an \gls{iot} ecosystem. The orchestrator selects resources and deploys the services according to security, reliability, and efficiency requirements. This approach enables an overall view of the whole environment, reducing costs and improving the user experience. Thus, orchestration allows the creation of more flexible and scalable services, reducing the probability of failure correlation between application components. 