\documentclass[journal]{IEEEtran}

\usepackage{amsmath}
\usepackage{booktabs}
\usepackage[table,xcdraw]{xcolor}
\usepackage{longtable}
\usepackage{adjustbox}
\usepackage{makecell}
\usepackage{subfigure}
\usepackage{lipsum}
\usepackage{pifont}
%\usepackage{tabularx}
\usepackage{ctable}
\usepackage{pbox}
\usepackage[T1,hyphens]{url}  
\usepackage{multirow}
\usepackage{makecell}
\usepackage{hhline}
\usepackage{wasysym}
\usepackage{amssymb}
\usepackage[acronym,smallcaps]{glossaries}
\usepackage[normalem]{ulem}

\usepackage{makeidx}

%\usepackage[table,x11names]{xcolor}
%\usepackage[table]{xcolor}
%\usepackage{tabularx,colortbl}
%\usepackage{color}

%\usepackage[colorlinks=true,citecolor=green,urlcolor=SteelBlue4,linkcolor=Firebrick4,hyperindex,breaklinks]{hyperref}
\usepackage[urlcolor=SteelBlue4,linkcolor=Firebrick4,hyperindex,breaklinks]{hyperref}
\usepackage[hyphenbreaks]{breakurl}

\useunder{\uline}{\ul}{}

% ========== misc ==========
\newacronym{ai}{AI}{Artificial Intelligent}
\newacronym{alto}{ALTO}{Application-Layer Traffic Optimization}
\newacronym{wan}{WAN}{Wide Area Network}
\newacronym{asic}{ASIC}{Application Specific Integrated Circuits}
\newacronym{osi}{OSI}{Open Source Initiative}
\newacronym{bsd}{BSD}{Berkeley Software Distribution}
\newacronym{sriov}{SR-IOV}{Single Root I/O Virtualization}
\newacronym{ict}{ICT}{Information and Communication Technology}

\newacronym{epa}{EPA}{Enhanced Platform Awareness}
\newacronym{numa}{NUMA}{Non-Uniform Memory Access}

\newacronym{enb}{eNB}{Evolved NodeB}
\newacronym{rsu}{RSU}{Roadside Unit}

\newacronym{escape}{ESCAPE}{Extensible Service ChAin Prototyping
Environment}

\newacronym{pox}{POX}{Python-based software-defined networking}

\newacronym{iot}{IoT}{Internet of Things}

\newacronym{gui}{GUI}{Graphical User Interface}
\newacronym{odl}{ODL}{OpenDaylight}
\newacronym{os}{OS}{OpenStack}

\newacronym{dpdk}{DPDK}{Data Plane Development Kit}
\newacronym{xdpd}{xDPd}{eXtensible OpenFlow DataPath daemon}

\newacronym{lsi}{LSI}{Logical Switch Instance}

\newacronym{cots}{COTS}{Commodity Off the Shelf}

\newacronym{capex}{CAPEX}{Capital Expenditure}
\newacronym{opex}{OPEX}{Operational Expenditure}

\newacronym{it}{IT}{Information Technologies}
\newacronym{ha}{HA}{High-Availability}
\newacronym{nsc}{NSC}{Network Service Chain}
\newcommand{\NSCs}{\glspl{nsc}\xspace}

\newacronym{em}{EM}{Element Management}
\newacronym{ems}{EMS}{Element Management Systems}
\newacronym{oss}{OSS}{Operation Support Systems}
\newacronym{bss}{BSS}{Business Support Systems}
\newacronym{oam}{OAM}{Operations and Management}

\newacronym[longplural=Points of Presence]{pop}{PoP}{Point of Presence} 

\newacronym{abno}{ABNO}{Application-based Network Operations}
\newacronym{cord}{CORD}{Central Office Re-architected as a Datacenter}

\newacronym{api}{API}{Application Programming Interface} 
\newcommand{\APIs}{\glspl{api}\xspace}
\newacronym{cli}{CLI}{Command Line Interface} 

\newacronym{sdo}{SDO}{Standards Developing Organization} 
\newcommand{\SDOs}{\glspl{sdo}\xspace}


\newacronym{hw}{HW}{hardware} 

\newacronym{ott}{OTT}{Over The Top}

\newacronym{sp}{SP}{Service Provider}


\newacronym{xacml}{XACML}{eXtensible Access Control Markup Language}
\newacronym{pap}{PAP}{Policy Administration Point}
\newacronym{pep}{PEP}{Policy Enforcement Point}
\newacronym{pdp}{PDP}{Policy Decision Point}
\newacronym{pip}{PIP}{Policy Information Point}

\newacronym{nat}{NAT}{Network Address Translation}
\newacronym{dhcp}{DHCP}{Dynamic Host Configuration Protocol}
\newacronym{bng}{BNG}{Broadband Network Gateway}
\newacronym{fw}{FW}{Firewall}
\newacronym{vfw}{vFW}{virtual Firewall}
\newacronym{lb}{LB}{Load Balancer}

\newacronym{slm}{SLM}{Service Level Measurement}

\newacronym{sla}{SLA}{Service Level Agreement}

\newacronym{kqi}{KQI}{Key Quality Indicator}

\newacronym{kpi}{KPI}{Key Performance Indicator}
\newcommand{\KPIs}{\glspl{kpi}\xspace}

\newacronym{qos}{QoS}{Quality of Service}

\newacronym{vm}{VM}{Virtual Machine}
\newcommand{\VM}{\gls{vm}\xspace}
\newcommand{\VMs}{\glspl{vm}\xspace}

\newacronym{dpi}{DPI}{Deep Packet Inspection}
\newacronym{vdpi}{vDPI}{virtual Deep Packet Inspection}

\newacronym{cp}{CP}{Control Plane}
\newacronym{dp}{DP}{Data Plane}

\newacronym{soa}{SOA}{Service-Oriented Architecture}

%%%% Organizations %%%%
\newacronym{ietf}{IETF}{Internet Engineering Task Force}
\newacronym{irtf}{IRTF}{Internet Research Task Force}
\newacronym{itu}{ITU}{International Telecommunication Union}
\newacronym{mef}{MEF}{Metro Ethernet Forum}
\newacronym{nist}{NIST}{National Institute of Standards and Technology}
\newacronym{3gpp}{3GPP}{3rd Generation Partnership Project}


\newacronym{fi}{FI}{Future Internet}

\newacronym{do}{DO}{Domain Orchestrator}

\newacronym[longplural=Multi-Domain Orchestrators]{mdo}{MDO}{Multi-Domain Orchestrator}
\newcommand{\MDOs}{\glspl{mdo}\xspace}

% ========== Cloud ==========
\newacronym{saas}{S$aa$S}{Software as a Service}
\newacronym{iaas}{I$aa$S}{Infrastructure as a Service}
\newacronym{paas}{P$aa$S}{Platform as a Service}
\newacronym{naas}{N$aa$S}{Network as a Service}
\newacronym{nfviaas}{NFVI$aa$S}{\gls{nfvi} as a Service}
\newacronym{vnfaas}{VNF$aa$S}{\gls{vnf} as a Service}
\newacronym{slaas}{Sl$aa$S}{Slice as a Service}
\newacronym{vnpaas}{VNP$aa$S}{Virtual Network Platform as a Service}
\newacronym{xaas}{X$aa$S}{Anything as a Service}
%\newacronym{do}{DO}{Domain Orchestrator}
%\newacronym{do}{DO}{Domain Orchestrator}

% ==========  ETSI ==========
\newacronym{etsi}{ETSI}{European Telecommunications Standards Institute}
\newacronym{isg}{ISG}{Industry Specification Group}

\newacronym[longplural=Network Function
Virtualization]{nfv}{NFV}{Network Function Virtualization}
\newcommand{\NFV}{\gls{nfv}\xspace}



\newacronym{vnf}{VNF}{Virtualized Network Function}
\newcommand{\VNF}{\gls{vnf}\xspace}
\newcommand{\VNFs}{\glspl{vnf}\xspace}
\newacronym{pnf}{PNF}{Physical Network Function}
\newcommand{\PNFs}{\glspl{pnf}\xspace}
\newacronym{vdu}{VDU}{Virtual Deployment Unit}

\newacronym{vnfd}{VNFD}{Virtualized Network Function Descriptor}

\newacronym{nfvo}{NFVO}{Network Function Virtualization Orchestrator}
\newacronym{vnfm}{VNFM}{VNF Manager}
\newacronym{vim}{VIM}{Virtualized Infrastructure Manager}

\newcommand{\NFVO}{\gls{nfvo}\xspace}
\newcommand{\VNFM}{\gls{vnfm}\xspace}
\newcommand{\VIM}{\gls{vim}\xspace} 
\newcommand{\VIMs}{\glspl{vim}\xspace} 

\newacronym{nfvi}{NFVI}{NFV Infrastructure}


\newacronym{nf}{NF}{Network Function}
\newcommand{\NF}{\gls{nf}\xspace}
\newcommand{\NFs}{\glspl{nf}\xspace}

\newacronym{vl}{VL}{Virtual Link}
\newcommand{\VL}{\gls{vl}\xspace}
\newcommand{\VLs}{\glspl{vl}\xspace}

\newacronym{vnffg}{VNF-FG}{\gls{vnf} Forwarding Graph}

\newacronym{mano}{MANO}{Management and Orchestration}

\newacronym{nfvmano}{NFV-MANO}{NFV-Management and Orchestration}

\newacronym{nso}{NSO}{Network Service Orchestration}

\newacronym{so}{SO}{Service Orchestrator}
\newacronym{lo}{LO}{Lifecycle Orchestrator}

\newacronym{tosca}{TOSCA}{Topology and Orchestration Specification for Cloud Applications}

% ========== SDN ==========

\newacronym{onf}{ONF}{Open Networking Foundation}

\newacronym{sdn}{SDN}{Software Defined Networking}
\newcommand{\SDN}{\gls{sdn}\xspace}

\newacronym[longplural=Open Flow Switches,shortplural=OFSes]{ofs}{OFS}{Open Flow Switch}
\newacronym[longplural=OpenvSwitches,shortplural=OVSes]{ovs}{OVS}{OpenvSwitch}

\newacronym{fe}{FE}{Forwarding Element}
\newacronym{ne}{NE}{Network Element}
\newacronym{nos}{NOS}{Network Operating System}

\newacronym[longplural=Southbound Interfaces,shortplural=SBIs]{sbi}{SBI}{Southbound Interface}
\newacronym[longplural=Northbound Interfaces,shortplural=NBIs]{nbi}{SBI}{Northbound Interface}

\newacronym{onos}{ONOS}{Open Network Operating System}
\newacronym{e2e}{E2E}{End-to-End}

% ========== SFC ==========
\newacronym{ns}{NS}{Network Service}

\newacronym{nfc}{NFC}{Network Function Chaining}

\newacronym{sfc}{SFC}{Service Function Chaining}

\newacronym{sf}{SF}{Service Function}

\newacronym{vcpe}{vCPE}{virtual Customer Premises Equipment}

% ========== UNIFY ==========

\newacronym{nffg}{NF-FG}{Network Function Forwarding Graph}
\newcommand{\NFFG}{\gls{nffg}\xspace}
\newcommand{\NFFGs}{\glspl{nffg}\xspace}

\newacronym{nfib}{NF-IB}{Network Function Information Base}

\newacronym{un}{UN}{Universal Node}

\newacronym{sg}{SG}{Service Graph\glsadd{sgg}}
\newcommand{\SG}{\gls{sg}\xspace}

\newacronym{rg}{RG}{Resource Graph\glsadd{rgg}}

\newacronym{sm}{SM}{Service Management}
\newacronym{af}{AF}{Adaptation Function}

\newacronym{nsor}{NSO}{Network Service Orchestrator}

\newacronym{ro}{RO}{Resource Orchestrator}
\newcommand{\RO}{\gls{ro}\xspace}
\newacronym{ca}{CA}{Controller Adapter}
\newcommand{\CA}{\gls{ca}\xspace}

\newacronym{sl}{SL}{Service Layer}
\newacronym{ol}{OL}{Orchestration Layer}
\newacronym{cas}{CAS}{Controller Adaptation Sublayer}
\newacronym{ros}{ROS}{Resource Orchestration Sublayer}
\newacronym{il}{IL}{Infrastructure Layer}
\newacronym{mp}{MP}{Management Plane}
\newacronym{nfs}{NFS}{Network Functions System}
\newcommand{\NFS}{\gls{nfs}\xspace}
\newcommand{\SL}{\gls{sl}\xspace}
\newcommand{\OL}{\gls{ol}\xspace}
\newcommand{\IL}{\gls{il}\xspace}

\newacronym{dov}{DoV}{Domain Virtualizer}
\newacronym{raf}{CA-RAF}{Controller Adapter-Resource Abstraction Function}
\newacronym{drdb}{DRDB}{Domain Resource Database}
\newacronym{pef}{PEf}{Policy Enforcement}
\newacronym{vcm}{VCM}{Virtual Context Manager}

\newacronym{sap}{SAP}{Service Access Point}
\newcommand{\SAP}{\gls{sap}\xspace}

%================ solutions/projects ====================
\newacronym{osm}{OSM}{Open Source MANO}
\newacronym{openo}{Open-O}{Open-Orchestrator}
\newacronym{5gex}{5G-Ex}{5G-Exchange}
\newacronym{aria}{ARIA}{Agile Reference Implementation of Automation}
\newacronym{onap}{ONAP}{Open Network Automation Platform}
\newacronym{5gppp}{5G-PPP}{5G Infrastructure Public Private Partnership}
\newacronym{sonata}{SONATA}{Service Programming and Orchestration for Virtualized Software Networks}

% \newglossaryentry{epg}{name=\acrfull{ep}, description={A service
%     provider's physical or logical port, which represents customers'
%     point of presence, access to internal services or exchange points
%     to other providers.  SAP definitions are included into the
%     \gls{u-res-srv} virtualization.}}
% \newacronym{ep}{EP}{End Point\glsadd{epg}}

\newacronym{lso}{LSO}{Lifecycle Service Orchestration}

\newacronym{dc}{DC}{Data Center}
\newacronym{cn}{CN}{Compute Node}

\newacronym[longplural=Big Switch and
Big Software Nodes,shortplural=BiS-BiS Nodes]{bsbs}{BiS-BiS}{Big
  Switch with Big Software}

% ========== SP-DevOps ==========
\newacronym{spdevops}{SP-DevOps}{Service Provider DevOps}

\newacronym{op}{OP}{Observability Point}

\newacronym{mf}{MF}{Monitoring Function}

\newacronym{rpc}{RPC}{Remote Procedure Call}
\newacronym{rest}{REST}{REpresentational State Transfer}


% ========== IDS app example ==========

\newacronym[longplural=Intrusion Detection Systems]{ids}{IDS}{Intrusion Detection System}
\newcommand{\IDSs}{\glspl{IDSs}\xspace}

\newacronym{idsc}{IDSC}{\gls{ids} Control}
\newacronym{eids}{E-IDS}{Elastic \gls{ids}}
\newacronym{eidsc}{E-IDSC}{Elastic \gls{idsc}}

% ========== VBaaS ==========
\newacronym{vbaas}{VBaaS}{\gls{vnf} Benchmarking as a Service}

\newacronym{ib}{IB}{Information Base}

\newacronym{actn}{ACTN}{Abstraction and Control of Transport Networks}
\newacronym{opnfv}{OPNFV}{Open Platform for NFV}
\newacronym{ims}{IMS}{IP Multimedia Subsystem}
\newacronym{cdn}{CDN}{Content Delivery Network}

\newacronym{sdwan}{SD-WAN}{Software Defined Wide Area Network}

\newacronym{ngmn}{NGMN}{Next Generation Mobile Networks}

\newacronym{lte}{LTE}{Long Term Evolution}

\newacronym{oasis}{OASIS}{Organization for the Advancement of Structured Information Standards}

\newacronym{pce}{PCE}{Path Computation Element}
\newacronym{mpls}{MPLS}{Multi-Protocol Label Switching}
\newacronym{lsp}{LSP}{Label Switching Path}

\newacronym{wg}{WG}{Working Group}

\newacronym{nfvrg}{NFVRG}{\gls{nfv} Research Group}
\newacronym{sdnrg}{SDNRG}{\gls{sdn} Research Group}


%%% Local Variables: 
%%% mode: latex
%%% TeX-master: "bare_conf"
%%% End: 


\hyphenation{op-tical net-works semi-conduc-tor}


\begin{document}

\title{Network Service Orchestration: A Survey}

\author{Nathan~F.~Saraiva~de~Sousa, Danny~A.~Lachos~Perez, Raphael~V.~Rosa, \\Mateus~A.~S.~Santos, and~Christian~Esteve~Rothenberg
\thanks{This work is under submission for peer review. Until then, and even after an eventual publication, the authors are most welcome for any feedback to improve the work and turn the \textit{github} and \textit{arxiv} versions of this publication a ``living document'' driven by community contributions as NSO evolves.  
Do not hesitate to contact the authors and/or submit \textit{github} pull requests or issues: 
https://github.com/intrig-unicamp/publications/tree/master/NSO-Survey.}
\thanks{N. Saraiva, D. Lachos, R. Rosa, and C. Rothenberg are with Department of Computer Engineering and Industrial Automation, Electrical and Computer Engineering, University of Campinas (UNICAMP), Campinas, SP, Brazil.}
\thanks{M. Santos is with Ericsson Research, Indaiatuba, SP, Brazil.}
}

\markboth{IEEE COMMUNICATIONS SURVEYS \& TUTORIALS}
{Shell \MakeLowercase{\textit{et al.}}: Bare Demo of IEEEtran.cls for IEEE Journals}

\maketitle

\begin{abstract}
Business models of network service providers are undergoing an evolving transformation fueled by vertical customer demands and technological advances such as 5G, Software Defined Networking (SDN), and Network Function Virtualization (NFV). 
Emerging scenarios call for agile network services consuming network, storage, and compute resources across heterogeneous infrastructures and administrative domains. 
Coordinating resource control and service creation across interconnected domains and diverse technologies becomes a grand  challenge. Research and development efforts are being devoted on enabling orchestration processes to  automate, coordinate, and manage the deployment and operation of network services. 
In this survey, we delve into the topic of Network Service Orchestration (NSO) by reviewing the historical background, relevant   research projects, enabling technologies, and standardization activities. We define key concepts and propose a taxonomy of NSO approaches and solutions to pave the way to the understanding  of the diverse ongoing efforts towards the realization of multiple NSO application scenarios. Based on the analysis of the state of affairs, we finalize by discussing a series of open challenges and research opportunities, altogether contributing to a timely and comprehensive survey on the vibrant and strategic topic of network service orchestration.
\end{abstract}

\begin{IEEEkeywords}
Network Service Orchestration (NSO), SDN, NFV, multi-domain, orchestration, virtualization, lifecycle management.
\end{IEEEkeywords}

\IEEEpeerreviewmaketitle

\section{Introduction}

Telecommunication  infrastructures consist of a myriad of technologies from specialized domains such as radio,  access, transport, core and (virtualized) data center networks. Designing, deploying and operating end-to-end services are commonly   manual and long processes performed via traditional \gls{oss} resulting in long lead times (weeks or months) until effective service delivery~\cite{BluePlanet2017ProductsOrchestration}. Moreover, the involved workflows are commonly hampered by built-in hazards of infrastructures strongly coupled to physical topologies and hardware-specific constraints.

\begin{figure}[t!]
  \centering
  \includegraphics[scale=.75]{Figures/01_Introduction/intro}
    \caption{Context and scope of Network Service Orchestration.}
    \label{intro}
\end{figure}


Technological advances under the flags of \gls{sdn} \cite{surveySDN} and \gls{nfv} \cite{Mijumbi2016NetworkChallenges} bring new ways in which network operators can create, deploy, and manage their services. \gls{sdn} and \gls{nfv}, as well cloud computing introduce new means for efficient and flexible utilization of their infrastructures through a software-centric service paradigm \cite{Sonkoly2014UNIFYingView}. However, to realize this paradigm, there is a need to model the end-to-end service and have the ability to abstract and automate the control of physical and virtual resources delivering the service. The coordinated set of activities behind such process is commonly referred to as \textit{orchestration}. In general, orchestration refers to the idea of automatically selecting and controlling multiple resources, services, and systems to meet certain objectives (e.g., a customer requesting a specific network service). Altogether, the process shall be timely, consistent, secure, and lead to cost reduction due to automation and virtualization. We refer to \gls{nso} as the automated management and control processes involved in services deployment and operations  performed mainly by telecommunication operators and service providers, involving different types of resources and potentially multiple providers, as illustrated in Figure~\ref{intro}.



\begin{figure*}[t!]
  \centering
  \includegraphics[scale=.6]{Figures/01_Introduction/org}
    \caption{Overview of the organization of this survey on NSO.}
    \label{org}
\end{figure*}

\gls{nso} is responsible for decoupling the high-level service layer (e.g., applications, service  slices, \gls{oss}) from the underlying management and resources layers (e.g., controllers, \gls{ems}, \gls{vim}), providing agility, enabling innovative service, optimizing resources, and altogether delivering a more flexible infrastructure for tailored services delivery. To this end, \gls{nso}  defines the interaction with (chains of) network functions in underlying technologies and infrastructures through adequate abstractions and a unifying pane glass for service definition and operation. For example, NSO may connect traditional OSS/BSS to network functions running in virtualized infrastructures. As depicted by the hourglass shape in Figure~\ref{orch}, the  significance of NSO as the inter-working glue resembles IP in the network protocol stack. %Besides, \gls{nso} gives service providers further control of their network services and enables developers to create new services and functions.  


\begin{figure}[t!]
  \centering
  \includegraphics[scale=.2]{Figures/01_Introduction/orchestrator.pdf}
    \caption{Strategic role of the \gls{nso} as the glue between the actual services and the underlying management of resources.}
    \label{orch}
\end{figure}

As today, broad understanding and practical definitions of \gls{nso} are still missing -- not only across but also inside networking communities. The maturity of ongoing efforts varies largely with the overall technical approach being very much fragmented and showing little consolidation around an overarching notion of network service orchestration. %Generally, its scope is associated to network evolution.  

%%%%%% The orchestration and management piece is a higher layer which can glue the network, NFV, and many other pieces all together to create an automated business-driven platform. Orchestration and virtual management platform interact with the network control layer to create and provision the networking requirements, such as policies, network slices, traffic redirection, and service insertion for workloads. Source: EdX Course %%%%%%

%Many stakeholders are involved in the development and standardization of enabling technologies for network softwarization and their embodiment into next generation networks (e.g. 5G). The ecosystem includes \glspl{sdo}, as well as industry groups, open source projects, foundations, diverse user-lead groups, and so on. Examples of these players include \gls{etsi}, \gls{mef}, \gls{oasis}, Linux Foundation, and \gls{onf}. Similarly, in recent years, many (academic and industrial) research and (commercial) development efforts in orchestration, SDN and NFV have been concretized in a number of collaborative endeavors, for instance \gls{osm}~\cite{ETSIOpenMANO}, OpenStack~\cite{Foundationb}, \gls{onap}~\cite{onap}, \gls{5gex}~\cite{Guerzoni2016}, \gls{cord}~\cite{ON.LABOpenCORD}, etc.

%%%%%% ENHANCE
The main objective of this survey is to provide a comprehensive understanding of the research, standardization, and software development efforts around the overcharged term of  \acrlong{nso}. We present an in-depth and up-to-date study on network service orchestration covering some historical background and context, enabling technologies, standardization activities, actual solutions, open challenges, and research opportunities. We propose a taxonomy of the main characteristics and features of NSO approaches. We also make the mapping of the \gls{nso} primary characteristics and technical implementations to current open source platforms and research projects.    

Throughout the survey, we distinguish between two types of domains. First, \textit{administrative domains}, which map to different organizations and therefore may exist within a single service provider or cover a set of service providers. In one administrative domain, multiple \textit{technology domains} can exist based on the type of technology in scope, for example, Cloud, \gls{sdn}, \gls{nfv}, or Legacy. 
Broadly speaking, we refer to \gls{nso} as the automated coordination of resources and services embracing both single-domain and multi-domain footprints.  
%Network service orchestration aims at addressing operational hazards of service providers.

Figure~\ref{mdo} presents a generic high-level reference model for multi-domain Network Service Orchestration, featuring a \gls{mdo} per administrative realm and including the notion of a Marketplace for business interactions. 
\glspl{mdo} coordinate resources and services in a multiple administrative domain scope covering multiple technology domains~\cite{5GPPPArchitectureWorkingGroup2016ViewArchitecture}. 
The exchange of information, resources, and services themselves are essential components of an end-to-end network service delivery.  The \gls{mdo} exposes the available services to the marketplace allowing service providers to sell network services directly to their customers or other providers under various possible resources consumption models (e.g., trading resources from each other). 
The \gls{mdo} can be seen as a single element with a possible split into two functional components: \gls{so} and \gls{ro}. The \gls{so} orchestrates high-level services while the \gls{ro} is responsible for managing resource and orchestrating workflows across technology domains. 
The \glspl{do} perform orchestration in each local domain acting on the underlying infrastructures and exposing resources and network functions northbound to the \gls{mdo}. 

%%%%%% Related Work
\noindent \textbf{Related work.} Several works address the theme of orchestration in different scopes including clouding computing \cite{Weerasiri2017}, \gls{sdn}~\cite{Jarraya2014},~\cite{surveySDN}, and \gls{nfv}~\cite{YongLi2015Software-DefinedSurvey},~\cite{Mijumbi2016NetworkChallenges},~\cite{Bhamare2016}. In~\cite{Weerasiri2017}, for example, the authors propose an taxonomy and survey of cloud resource orchestration techniques. However, its scope is limited to cloud resources. The work of Rotsos~et~al.~\cite{Rotsos2017NetworkSurvey} is the first notable attempt to survey the realm of network service orchestration. The authors provide an analysis of diverse standardization activities around \gls{nso} from an operator perspective. The article follows a top-down approach, defining terminologies, requirements, and objectives of a network service orchestrator. In contrast, our definition and approach to \gls{nso} are distinct than previous works. We follow a systems-oriented and broadly generic approach,  where \gls{nso} encompasses high-level services as defined by telecommunications operators along business and technological operations for network service instantiation and run-time operation. Most significantly, we feature 150+ references providing a broader scope covering:
\begin{itemize}
\item Historical review of the overloaded term orchestration;
\item How several communities approach orchestration in different areas;
\item Comprehensive definition of \gls{nso} clarifying aspects such as the relation between orchestration, management, and automation, and the core NSO functions;
\item Taxonomy to present the main aspects of any NSO solution;
\item Up-to-date review of ongoing standardization activities;
\item Overview of relevant research projects and software frameworks;
\end{itemize}

\begin{figure*}[th]
  \centering
  \includegraphics[scale=.6]{Figures/01_Introduction/nso}
    \caption{High-level reference model to illustrate the scope of \acrfull{nso} in single-domain and multi-domain environment. The \gls{nso}  need to have an overview of entire environment to compose the service mainly if it uses resources of different domains.}
    \label{mdo}
\end{figure*}
%a clear definition of \gls{nso}, standardization outcomes, research projects, related frameworks, application scenarios, and challenges.

%The \gls{mdo} exposes the available services to the marketplace. The marketplace allows Service Providers to purchase network services from \acrlongpl{mdo} and offers them to their customers even though the services are composed of resources from other domains.

%Marketplace allows SPs to purchase VNFs from software developers in order to compose NS and offer them to their customers, including SLA management, accounting and billing features and the corresponding interfaces with Service orchestrator.

%The main objective of this paper is to survey on network service orchestration in current context and both single and multi-domain environments. It aims also provide a comprehensive understanding of this new scenarios, its related technologies, as well as the main issues that related with the standardization process. 

%We propose the best of our vision on NSO in this work.


\noindent \textbf{Survey Organization.}  The survey is organized as depicted in Figure~\ref{org}. Section~\ref{sec:background} presents essential background and key technologies related to network service orchestration: Cloud computing, \gls{sdn}, \gls{nfv}, historical overview of orchestration, and the relationship between all mentioned technologies. Section~\ref{sec:scneario} outlines four potential scenarios to illustrate the \gls{nso} in practice. Concepts, functions, scope, and an NSO taxonomy split into seven key aspects are presented in Section~\ref{sec:nso}. Section~\ref{sec:stand} focuses on the standardization outcomes produced by nine important organizations,   whereas Section~\ref{sec:project} covers six major research projects around \gls{nso}. Section~\ref{sec:proj} provides an overview of ten open source solutions and some commercial initiatives.  The discussion in Section~\ref{sec:challenge} points to six groups of open challenges and research opportunities. Finally, Section~\ref{sec:Conclusion} concludes the survey.

%%%%%% Main contributions %%%%%%
% Historical review of term orchestration
% Overview about definition of orchestration in various organizations and works
% Definition clear about Network Service Orchestration 
%%% Relation between orch, management and automation
%%% Its different functions: Service Orch, Lifecycle Orch and Resource Orch.
%%% Difference and definition about Lifecycle and workflow.  
% Taxonomy
% Main outcomes of standardization entities related to NSO
% Approach research projects and the aspects NSO that they address 
% Point the solutions that implement concepts of NSO.
\section{Background}
\label{sec:background}

\gls{nso} foundations can be rooted back to three enabling technologies, namely Cloud Computing, SDN, and NFV. This section provides a brief background on these topics and their relationships to NSO, in addition to a short historical review of the term ``orchestration''.

\subsection{Cloud Computing}
Cloud computing is a model for providing resource virtualization (e.g., networks, servers, storage, and services) with high flexibility, cost efficiency, and centralized management~\cite{Le2016SurveyNetworks}. The cloud computing service models are generally categorized in \gls{iaas}, \gls{paas}, and \gls{saas} which offer, respectively,  virtual resources (compute, storage, and network), software and development platforms (provided by the cloud infrastructure), and Internet-based applications (hosted on the cloud)~\cite{bele2018empirical}.

%Cloud computing is a model for enabling ubiquitous, convenient, on-demand network access to a shared pool of configurable computing resources (e.g., networks, storage, and services) that can be rapidly and automatically provisioned and released with minimal effort~\cite{Mell2011TheTechnology}. Thereby, the resources are traded on demand, that is, the customer only pays what is used. Cloud computing becomes one of the relevant technology for the 5G networks mainly because it provides high data rate, high mobility, and centralized management \cite{Le2016SurveyNetworks}.

%The service models of cloud computing  are generally categorized into three classes: \gls{saas}, \gls{paas}, and \gls{iaas}. In a cloud \gls{iaas}, the infrastructure is offered as a service to the customer. Each customer can have its virtual resources, such as compute, storage, and network. 
%\gls{saas} includes applications such as Facebook, Google Apps, Twitter, and Microsoſt Office 365.

%\gls{paas} provides services according to a user’s applications without installing or configuring the operating system. The customers can develop and deploy their applications in the same development environment. The \gls{paas} model includes services such as Microsoft Azure, Google App Engine, RedHat OpenShift, and Amazon Elastic Beanstalk.  

%In \gls{saas}, in turn, the customer is able to use the providers' applications running on a cloud infrastructure~\cite{Mijumbi2016NetworkChallenges}. The softwares are maintained and managed by a cloud provider. \gls{iaas} includes applications, for example, OpenStack, CloudStack, Amazon CloudFormation, and Google Compute Engine.

In a cloud environment, the notion of orchestration has also been used for integrating basic services~\cite{Vouk2008CloudImplementations}. The Orchestration in the cloud involves dynamically deploying, managing and maintaining resource and services across multiple heterogeneous cloud platforms in order to meet the needs of clients. 
%This procedure demands to automatize processes and create a workflow.
%However, this is not a simple task.

%The Cloud computing service layers are depicted in Figure~\ref{cloud}
% \begin{figure}[thpb]
%   \centering
%   \includegraphics[scale=.36]{Figures/02_Background/cloud2}
%     \caption{Cloud computing service layers.}
%     \label{cloud}
% \end{figure}


%Diference between cloud and nfv. See slide 60,68 NFV
%https://sdn.ieee.org/newsletter/september-2017/intent-based-management-and-orchestration-of-heterogeneous-openflow-iot-sdn-domains
\subsection{Software Defined Networking (SDN)}
%Software-defined networking
%Software Defined Networking
%Software Defined Network

%SDN~\cite{kreutz2015software} is an emerging networking paradigm that attempts to resolve the strongly vertical integration of current network environments. To do so, SDN proposes to decouple the control plane from the data plane (See Fig. x). With this new architecture , data forwarding equipments (e.g. routers and switches) become simple forwarding network elements whose control logic is provided by a external entity called SDN controller or network operating system (NOS). In the upper layer, software network customized applications 
%is used to provide  abstract the lower-level functionalities and  

\gls{sdn}~\cite{surveySDN} is an evolving networking paradigm that attempts to resolve the strongly vertical integration of current network environments. To this end, \gls{sdn} proposals decouple the control plane (i.e., control logic) from the data plane (i.e., data forwarding equipment). With this new architecture, routers and switches become simple forwarding network elements whose control logic is provided by an external entity called \gls{sdn} controller or \gls{nos}. 

\glspl{nbi} offered by a logically centralized \gls{sdn} controller allow different network applications (firewalls, routing, and resource orchestrators) to implement network control and operation logic. In addition, other type of high-level \acrshortpl{nbi} category are implemented as \gls{nos} management applications~\cite{Rotsos2017NetworkSurvey}. Examples of this category include Virtual Tenant Networks, \gls{alto}, and Intent-based networking (IBN).

%Service orchestrators, OSSes and other network applications can be developed on top of high-level \glspl{nbi} offered by a logically centralized \gls{sdn} controller. Indeed, \acrshortpl{nbi} are crucial components to control and monitor the network services orchestration.

%The logically centralized \gls{sdn} controller acts in spirit of computer operating systems that provide a high-level abstraction for the management of computer resources (e.g., hard drive, CPU, memory) by playing the network operating system role for network management~\cite{gude2008nox}. As such, it provides a set of services (base network services, management, orchestration) and common interfaces (North/South/East/West) to developers who can implement different control applications and improve manageability of networks. Moreover, such interfaces are used within the \gls{mano} framework to deploy end-to-end connectivity. As today, the most popular open source \gls{sdn} controllers are \gls{onos}~\cite{ON.LABONOSScale-out.} and OpendayLight~\cite{LinuxFoundationOpenDaylight}.


%Service orchestrators, OSSes and other network applications can be developed on top of high-level \glspl{nbi} offered by a \gls{sdn} controller. %Indeed, \acrshortpl{nbi} are crucial components to control and monitor the network services orchestration.

%The \gls{sdn} controller creates an abstract network view while hiding details of the underlying physical or virtual infrastructure. Running on the top of the \gls{sdn} controller, software network applications can perform not only traditional functionalities such as routing, load balancing, classification~\cite{6965141}, or \acrlongpl{ids}, but also propose novel use cases such as service  orchestration across multi-domain and multi-technology in 5G networks~\cite{Bernardos20155GInfrastructures}. Those applications, together with others industry and academy initiatives towards flexible network services over programmable resources are, among the main drivers of \gls{sdn}.

%Service orchestrators, OSSes and other network applications can be developed on top of high-level \glspl{nbi} offered by a \gls{sdn} controller. Indeed, \acrshortpl{nbi} are crucial components to control and monitor the network services orchestration. Unlike \acrshortpl{sbi}, where Openflow is a well-known \gls{sdn} standard protocol, \acrshortpl{nbi} are still an open issue with different controllers offering a variety of \acrshortpl{nbi} (e.g., RESTful APIs~\cite{richardson2008restful}, NVP NBAPI~\cite{onix},~\cite{koponen2014network}, SDMN API~\cite{pentikousis2013mobileflow}, etc.). In addition, other type of high-level \acrshortpl{nbi} category are implemented as \gls{nos} management applications~\cite{Rotsos2017NetworkSurvey}. Examples of this category include Virtual Tenant Networks, ALTO, and Intent-based networking (IBN).

%The communication between the \gls{sdn} controller and the forwarding devices is done through \glspl{sbi}, which allow decoupling the control and data plane via open communication protocols (i.e. well-defined APIs). Different \gls{sdn} \glspl{sbi} can be considered (e.g., ForCES~\cite{Doria2010}, OVSDB~\cite{Davie2013RFCProtocol}, POF~\cite{song2013protocol}, etc.), with  OpenFlow~\cite{openFlow},~\cite{SDXCentral2014WhatAPIs} being the most widely accepted solution available in commercial and open source (hardware and software) devices.  %The Openflow protocol provides three major informations for the SDN controller: (i), (ii), (iii)
%are enabling  providing customized and optimized network services 
%%and Robert Szabo  
%Software defined networking is a computer networking approach that lets network administrators manage network services through abstraction of lower-level functionality. SDN decouples the control plane (controllers and/or Network Operating Systems) from the data plane (equipments to forwarding of data) \cite{c7}[rever].
%SDN addresses the challenge that the static architecture of traditional networks does not support the scalable, dynamic computing and storage needs of dynamic environments, e.g. datacenter and operator networks. It creates an abstract vision of network that enable fast innovation and networks more agile and flexible.

%Service orchestrators, OSSes and other network applications can be developed on top of high-level \glspl{nbi} offered by a \gls{sdn} controller. Indeed, \acrshortpl{nbi} are crucial components to control and monitor the network services orchestration. Unlike \acrshortpl{sbi}, where Openflow is a well-known \gls{sdn} standard protocol, \acrshortpl{nbi} are still an open issue with different controllers offering a variety of \acrshortpl{nbi} (e.g., RESTful APIs~\cite{richardson2008restful}, NVP NBAPI~\cite{onix},~\cite{koponen2014network}, SDMN API~\cite{pentikousis2013mobileflow}, etc.). In addition, other type of high-level \acrshortpl{nbi} category are implemented as \gls{nos} management applications~\cite{Rotsos2017NetworkSurvey}. Examples of this category include Virtual Tenant Networks, ALTO, and Intent-based networking (IBN).% However, initiatives such as the ONF's NBI working group are making good progress developing an intent-based interface~\cite{OpenNetworkingFoundation2017Intent:BLOG},~\cite{NetworkComputing2015SDNsEvolves} which is expected to become a common interface to applications and services.% An orchestrator uses NBIs and the VNF Manager () to 

%http://www.networkcomputing.com/networking/sdns-northbound-interface-evolves/562466230
%https://www.opennetworking.org/?p=1633&option=com_wordpress&Itemid=155
%https://www.sdxcentral.com/articles/contributed/intent-based-networking-seeks-network-effect-david-lenrow/2015/09/

%The logically centralized \gls{sdn} controller acts in spirit of computer operating systems that provide a high-level abstraction for the management of computer resources (e.g., hard drive, CPU, memory) by playing the network operating system role for network management~\cite{gude2008nox}. As such, it provides a set of services (base network services, management, orchestration) and common interfaces (North/South/East/West) to developers who can implement different control applications and improve manageability of networks. Moreover, such interfaces are used within the \gls{mano} framework to deploy end-to-end connectivity. As today, the most popular open source \gls{sdn} controllers are \gls{onos}~\cite{ON.LABONOSScale-out.} and OpendayLight~\cite{LinuxFoundationOpenDaylight}.

In SDN, the concept of orchestration is vital to automate network operations properly. SDN network domains need single-domain or multi-domain orchestration systems to coordinate end-to-end connectivity services through different network domains controlled by different SDN controller instances, which in turn must communicate directly with the physical network~\cite{SDNevolution}.

%which in turn must communicate directly with the physical network~\cite{SDNevolution}.

%A hybrid SDN model introduces 

%Different SDN (and traditional) solutions are integrated to provide multi-layer orchestration over multiple domains. For example, ESCAPE uses Click [], POX [], OpenDaylight [], and NETCONF [] 
%SDN solutions: OpenDaylight, OpenStack, OPNFV
% SDN survey

%A typical orchestration framework performs tasks such as set up service chains, map VNFs to resources, steer traffic according to chains’ policies, and provide real-time management information on running VNFs []. The steering traffic flows between services 

%Single/Multi-layer orchestration over multiple domains involve different tasks () 

%Network Functions Virtualization (NFV) is an effort to make telecommunication services and service components software-based as much as possible1. By this means, whole services or service elements can run in virtualized environment on a wide range of general purpose hardwares which makes service deployment, configuration and operation easier. Moreover, the usage of different types of resources (e.g., compute and storage) can be optimized in a more flexible way and several tools are available from the Cloud world. Besides packet processing tasks assigned to (physical or virtual) network functions, steering traffic flows between these service elements is an indispensable part of service provisioning and SDN enables to realize it efficiently

%SDN provides mechanisms to automated orchestration. 
%Automated orchestration by way of SDN poses 
%Different orchestration frameworks combine 
%CONTROLLER ... Nowadays, SDN is an well-know technology and two solutions are dominating the market: ONOS (Open Network Operating System) \cite{ON.LABONOSScale-out.} and OpenDayLight \cite{LinuxFoundationOpenDaylight}.



\subsection{Network Function Virtualization (NFV)}
\label{subsec:nfv}
% http://www.lightreading.com/automation/process-automation-tops-carriers-goals-for-nfv/d/d-id/732921
Traditionally, the telecommunication operators have based their networks on a built-in infrastructure strongly coupled to physical topologies and proprietary devices, resulting in network services constrained to the network topology and the physical location of the network appliances. As a consequence, it becomes hard for providers to offer new services with lower cost and more efficiency and agility \cite{Mijumbi2016NetworkChallenges}. \acrlong{nfv} has been proposed to solve these problems \cite{ETSI2012NetworkAction} and change the mode networks are designed and operated by taking a software-centric approached leveraging advances in virtualization technologies and general purpose processors.

According to \gls{etsi} \gls{isg} \gls{nfv} \cite{ETSIIndustrySpecificationGroupISGNFV2014NetworkNFV},  \acrlong{nfv} is responsible for separating network functions from the hardware and offering them through virtualized services, decomposed into \gls{vnf}, on general purpose servers. With the virtualization of the network functions, \gls{nfv} promises more flexible and faster network function deployment, as well as dynamic scaling of the \glspl{vnf} towards providing finer settings. In \gls{nfv} environment, new services do not require new hardware infrastructure, but simply the software installation, i.e., to create \glspl{vnf}.

Moreover, the \gls{nfv} can address \glspl{nf} in the most appropriate location, providing better user traffic performance. The network service can be decomposed in one or more \glspl{vnf}, and each one can be constituted in one or more \glspl{vm}. Each \gls{vnf} is described by a \gls{vnfd} which details the behavioral and deployment information of a \gls{vnf}.

\glspl{vnf} can be connected or combined as building blocks to offer a full-scale network communication service. This connection is known as service chain. Service chain provides logical connectivity between the virtual devices of \gls{nfv} architecture. It is worthwhile noting not only connectivity order importance, but also the logical environment interconnection with physical networks. 

Within the scope of the \gls{isg} \gls{nfv} \cite{ETSIIndustrySpecificationGroupISGNFV2014NetworkNFV}, service chain is defined as a graph of logical links connecting \glspl{nf} towards describing traffic flow between these network functions. This is equivalent to the \gls{sfc}~\cite{Halpern2015} defined by Service Function Chaining Working Group (IETF SFC WG) of the \gls{ietf}.  
An end-to-end network service may cover one or more \gls{nffg} which interconnect \glspl{nf} and end points.  Figure~\ref{nffg} describes two examples of end-to-end network services. The first (green line) is composed of \gls{vcpe} and \gls{vfw} \glspl{vnf} and two endpoints (A1 and A2). The second (red line) is composed of \gls{vcpe} and \gls{vdpi} \glspl{vnf} and two endpoints (B1 and B2). Note that \gls{nfv} allows sharing a multi-tenant \glspl{vnf} between different network services. 

\gls{etsi} has developed a reference architectural framework and specifications in support of NFV management and orchestration. The framework focuses on the support \gls{vnf} operation across different hypervisors and computing resources. It also covers the orchestration and lifecycle management of physical and virtual resources. According to~\cite{ETSIIndustrySpecificationGroupISGNFV2013NetworkFramework}, ``the framework is described at a functional level and it does not propose any specific implementation." Figure~\ref{mano} shows the \gls{etsi} \gls{nfv}-\acrfull{mano} architectural framework with their main functional blocks~\cite{ETSIIndustrySpecificationGroupISGNFV2014NetworkOptions}:
%as described in the \acrfull{mano} specification \cite{ETSIIndustrySpecificationGroupISGNFV2014NetworkOptions}. 
%The \gls{etsi} \gls{nfv} architectural framework is composed mainly of seven functional blocks :
 \\
 \noindent \textbf{Operation/ Business Support System (OSS/BSS)}: block responsible for operation and business applications that network service providers use to provision and operate their network services. It is not tightly integrated into the \gls{nfv}-\gls{mano} architecture.
 \\
\noindent \textbf{\gls{em}}: component responsible for the network management functions FCAPS (Fault, Configuration, Accounting, Performance, and Security) of a running \gls{vnf}.
\\
\noindent\textbf{\gls{vnf}}: functional block representing the Virtualised Network Function implemented on a physical server. For instance, Router \gls{vnf}, Switch \gls{vnf}, Firewall etc.
\\
\noindent \textbf{\gls{nfvi}}: representing all the hardware (compute, storage, and networking) and software components where \glspl{vnf} are deployed, managed and executed. 
\\
\noindent \textbf{\gls{nfvo}}: it is the primary component, in charge of the orchestration of \gls{nfvi} resources across multiple \glspl{vim} and lifecycle management of network services. 
\\
\noindent\textbf{\gls{vnfm}}: performs configuration and \gls{vnf} lifecycle management (e.g., instantiation, update, query, scaling, termination) on its domain.
\\
\noindent \textbf{\gls{vim}}: block that provides controlling and managing the \gls{nfvi} resources as well the interaction of a \gls{vnf} with hardware resources. For example, OpenStack as cloud platform and OpenDaylight and \gls{onos} as \gls{sdn} controllers.
 

\begin{figure}[t]
  \centering
  \includegraphics[scale=.45]{Figures/02_Background/ns}
    \caption{Example of two end-to-end network services composed of two \glspl{nf} each. NFV enables the reuse of \glspl{vnf}, e.g., \gls{vcpe}.}
    \label{nffg}
\end{figure}

\begin{figure}[t]
  \centering
  \includegraphics[scale=.4]{Figures/02_Background/MANO}
    \caption{The \gls{nfvmano} architectural framework. Adapted from \cite{ETSIIndustrySpecificationGroupISGNFV2014NetworkOptions}}
    \label{mano}
\end{figure}

The \gls{nfvmano} functional block performs all the virtualization-specific management, coordination, and automation tasks in the \gls{nfv} architecture including the components \gls{nfvo}, \gls{vnfm}, \gls{vim}, \gls{nfv} Service, \gls{vnf} Catalogue, NFV Instance, and \gls{nfvi} Resource. 

In the \gls{nfv} context, \gls{etsi} \gls{nfvmano} defines the orchestrator with two main functions including \textit{resources orchestration across multiple \glspl{vim}} and \textit{network service orchestration}~\cite{GSNFV-MAN001:2014}. Network service orchestration functions provided by the \gls{nfvo} are listed below:
\begin{itemize}
\item Management of Network Services templates and \gls{vnf} Packages. This includes validation of templates and packages with the objective of verifying the artifacts' authenticity and integrity. Besides, the software images are cataloged in involved \glspl{pop} using the support of \gls{vim}.
\item Network Service instantiation and management;
\item Management of the instantiation of \glspl{vnfm} and \glspl{vnf} (with support of \glspl{vnfm});
\item Validation and authorization of \gls{nfvi} resource requests from \gls{vnf} managers;
\item Management of network service instances topology;
\item Policy management related to affinity, scaling (auto or manual), fault tolerance, performance, and topology.
\end{itemize}

\gls{etsi} \gls{nfvo} functions regarding Resource Orchestration include: (\textit{i})  Orchestration of NFVI resources across multiple \glspl{vim}, (\textit{ii})  \gls{nfvi} resource management including compute, storage and network, and (\textit{iii}), collect usage information of \gls{nfvi} resources.
%\end{itemize}

The \gls{nfvmano} reference architecture is not  specific about \gls{sdn} in its architecture but  assumes that necessary transport infrastructure is already established and ready to be used. However, work at \gls{etsi} identifies use cases and the most common options for using SDN in an NFV architectural framework~\cite{ETSINetworkFramework}. The document also points to  proof of concepts and recommendations towards such integration work.
\cite{nfv-survey18} provides a recent in-depth survey on NFV state of affairs. 

%, from standardization to research. 
%Note that the \gls{etsi} only describes the functionalities of \gls{nfvmano} architecture and no defines regarding the technical details mainly to end-to-end network services. Moreover, there is no specification for the case where involves multiple networks with different technologies \cite{Katsalis2016Multi-DomainDirections}.

%Mateus: Does this subsection fit to background? 
%%%%%%%: Christian's Suggestion 
\subsection{Orchestration: Historical Overview}



%generic definition
The academic community and industry generally require some time to define the real meaning, reach and context of the concepts related to new technology trends as is the case with the term \textit{Orchestration}. 
The term orchestration is used in many different areas, such as multimedia, music, \gls{soa}, business processes, Cloud, \gls{sdn}, and, more recently, in \gls{nfv}.

%Maybe add about business process

From an end-user perspective, orchestration reminds a symphony orchestra where a set of instruments play together according to an arrangement. The music is arranged and split into small parts, after assigns to different musical instruments. When, who, and what will be played, as well as the conducting are essential parts towards achieving the desired effect. In next paragraphs, we identified the first works that use the orchestration in other areas. 

One of the first works in the \gls{ict} area that cites the term orchestration is \cite{Anderson1983} in 1983. It discusses that an autonomous system will require orchestration of the behavior of the entire system in order to obtain autonomy, interdependence and artificial intelligence. The authors in~\cite{Campbell1992} relate orchestration with the coordination and control of multiple media traffics. It distinguishes the orchestration from synchronization and defines an architecture where the orchestration acts in different layers. In the same scope, \cite{Robbins1997ImplementationArchitecture} relates the term to multimedia data, where orchestration is associated with multimedia presentation lifecycle management involving the coordination of stages that constitute all orchestration processes. 

The use of orchestration is also widely discussed in the scope of web services. In this context, orchestration and automation are considered separate processes. The work in~\cite{Peltz2003WebChoreography} defines orchestration like an executable process that can interact both internal and external services and must be dynamic, flexible, and adaptable to changes. It emphasizes that orchestration describe how web services can act with each other at the message level, including the business logic and execution order of the activities. 

The authors in~\cite{Grit2006} present the term orchestration in the context of virtual resource management. They define the orchestration as a process that involves all the necessary steps to map the application (running on a virtual machine) onto shared underlying infrastructure. 

%More recently in 2009, \cite{Galis2009ManagementInternet} provides an overview of 
Orchestration in the cloud environment is well-known and refers to locating, coordinating and selecting resources, including compute, storage and virtual networks to fulfill the desired requirements. The authors in \cite{Galis2009ManagementInternet} provide an overview of networking architecture definition for the \gls{fi} based on the concepts of cloud computing. One of the pillars for the \gls{fi} pointed out by the article is Orchestration. In the envisioned architecture, the orchestration function is to coordinate the integrated behavior and operations to dynamically adapt and optimize resources in response to changing context following business objectives and policies.

In the \gls{sdn} landscape, orchestration refers to an overarching function to manage and automate the network behavior~\cite{5984813}. %This means to govern all processes involved in the forwarding of network traffic. 
 More recently in 2012~\cite{ETSI2012NetworkAction}, orchestration has been generally related to \gls{nfv} environments mainly through its reference architecture and its \gls{nfv} Orchestrator component (more details in Subsection~\ref{subsec:nfv}).  

Currently, the scope of the orchestration has become broader and encompasses automation of the end-to-end network service lifecycle. According to \cite{MEF:Third:2015}, service orchestration refers to the programmatic control of underlying infrastructure including existing networks and enabling technologies, such as SDN and NFV.

%Orchestration in the cloud environment is well-known and refers to the locating, coordinating and selecting of resources, including compute, storage and virtual network, to fulfill the desired requirements \cite{Abosi2011}.   
%In short, the generalized orchestration means the capability to arrange, coordinate and manage systems, services, and resources in order to achieve an optimal outcome.

%%Orchestration is generally related to service automation in cloud~\cite{Abosi2011} and \gls{nfv} environments. In spite of that, its concepts are not clearly defined in the scope of \gls{nfv} yet \cite{Kuklinski2016DesignOrchestrators},~\cite{Alvizu2016AdvanceEra}. %Currently, this term is applied in network services deployment of telecommunication operators and service providers. 

From the existing and evolving definitions around orchestration presented, we can derive certain relationships between orchestration, automation, and management. Although the three terms are often lumped together, it is necessary an understanding of the differences between them as they are not the same thing. Automation describes a simple and technical task without the human intervention, for example, launching a web server, stopping a server. Management is responsible for maintaining and healthiness of infrastructure. Its role consists of activities such as alarms for event detection, monitoring, backups of critical systems, upgrades, and license management. Orchestration, in turn, is concerned with the execution of a workflow (processes) in the correct order. It controls the overall workflow process from starting the service until it ends with the objective to optimize and automate the network service deployment. 

Figure~\ref{diff} illustrates the relationship among orchestration, management, and automation. There is a certain hierarchic between them. The orchestration is a high-level plane, below the management, and in the bottom the automation. In our vision, the orchestration depends on tasks performed by management. Both management and orchestration are based on the use of automation in the execution of their tasks. Nevertheless, several activities are only performed by a certain function: optimization, for instance, cannot be achieved through simple automation. There is a difference between them, but, if they work together in the execution of processes, the services deployments will succeed with further accuracy.

\begin{figure}[t]
    \centering
    \includegraphics[scale=.45]{Figures/02_Background/OrchManaAut}
      \caption{Relationship among orchestration, management, and automation. Both orchestration and management use automation in their processes.}
      \label{diff}
\end{figure}

Based on all the above-mentioned background, in short, \gls{nso} is in charge of the full network service lifecycle to deliver end-to-end connectivity along additional services. To this end, orchestration is supported by advances in cloud computing, and technologies such as \gls{sdn} and \gls{nfv}, which offer the ability to reconfigure the network quickly as well as programming the forwarding and processing of the traffic. Figure~\ref{nso_rel} shows how \gls{nso}, \gls{nfv}, \gls{sdn}, and Cloud Computing work together. %Figure~\ref{nso_rel} aims at illustrating the relationships between \gls{nso}, \gls{nfv}, \gls{sdn}, and Cloud Computing.

Each one of these paradigms/technologies has different functions: high level orchestration for \gls{nso}, function programming for \gls{nfv}, networking programming for \gls{sdn},  and resource virtualization for cloud computing. Note that such technologies are complementary in order to provide complete management of the network services lifecycle. Although they have different functions, they share a common feature: \textit{orchestration}. They can work in an integrated pattern to offer advantages such as agility, cost reduction, automation, softwarization, and end-to-end connectivity, to enable novel services and applications such as 5G networks.

\begin{figure}[t]
  \centering
  \includegraphics[scale=.45]{Figures/02_Background/nso_rel.pdf}
    \caption{Illustration of relationships among \gls{nso}, \gls{nfv}, \gls{sdn}, and Cloud.}
    \label{nso_rel}
\end{figure}

Our goal in this subsection was to set the ground and identify the main areas in which the term orchestration is inserted and how it is approached at a high level. An overview of the term usage is illustrated in the timeline of Table~\ref{timeline}. The focus of this survey is to detail the orchestration in the context of the implementation and operation of network services by operators and service providers.

\begin{table}[!]
\small
\caption{Historical timeline of term orchestration }\vskip -1ex
\label{timeline}
\begin{tabular}{@{\,}r <{\hskip 2pt} !{\foo} >{\raggedright\arraybackslash}p{5cm}}
\addlinespace[3ex]
\toprule
1983 & Autonomous system~\cite{Anderson1983}\\[13.5pt]
1992 & Media Traffic~\cite{Campbell1992}\\[7.5pt]
1997 & Multimedia presentation lifecycle management~\cite{Robbins1997ImplementationArchitecture}\\[1pt]
2003 & Web Service~\cite{Peltz2003WebChoreography}\\[4.5pt]
2006 & Virtual resource management~\cite{Grit2006}\\[4.5pt]
2009 & Cloud computing~\cite{Galis2009ManagementInternet}\\[3pt]
2011 & Software Defining Network~\cite{5984813}\\[1.5pt]
2012 & Network Function Virtualization~\cite{ETSI2012NetworkAction}\\[2pt]
2015 & Lifecycle Service Orchestration~\cite{MEF:Third:2015}\\
\end{tabular}
\end{table}


%\subsection{Relationship of Cloud, SDN, NFV, and NSO}

%%%%%% Improve the text about relevance of NSO %%%%%%% DONE




% \begin{figure}[t]
%     \centering
%     \includegraphics[scale=.4]{Figures/02_Background/OrchManaAut}
%       \caption{Relationship among orchestration, management, and automation. Both orchestration and management use automation in their processes.}
%       \label{diff}
% \end{figure}

%There is a difference between them but if they work together in the execution of processes, the services deployments will be successful and with further accuracy.


%%%%
%In short, the generalized orchestration means the capability to arrange, coordinate and manage systems, services, and resources in order to achieve an optimal outcome.

%Difference between orchestration and automation and management.
%ITU-T Y.3111

\section{Network Service Orchestration}
\label{sec:nso}

\subsection{Towards a Practical Definition}
\label{sec:def}

We refer to the \acrfull{nso} when applied in the services deployment performed by telecommunication operators and service providers. We regard NSO not precisely as a unique technology but a concept to  understand network services in detail, relying on multiple technologies and paradigms to achieve such an overarching goal. In a nutshell, network service orchestration comprises the semantics of requested service, and thereby it coordinates specific actions in order to fulfill the service requirements and to manage its end-to-end lifecycle. 

The entire orchestration process proposed by NSO involves business and operations that go beyond the delivery of \textit{network services} as defined by \gls{etsi}. \gls{etsi} \gls{nfvmano} is a platform for management and orchestration required to provisioning \glspl{vnf} in an \gls{nfv} domain. The \gls{mano} is agnostic and thus has no insight of what is executed within a \gls{vnf}, restricting its responsibility and capability to the VNF instantiation and lifecycle management.

Based on Figure~\ref{mdo}, the \gls{mdo} understands the operating capabilities of the \gls{ns} in a broad sense. When a customer demands an \gls{ns}, firstly it requests the order to a service provider or telecommunication operator through Business-to-Business (B2B) interface or a trading platform we refer to as Marketplace. After that, the \gls{mdo} interacts with any \gls{mano} element or other elements (e.g., OSS/BSS, SDN Controllers, Analytic Systems)  to create the \gls{ns}. Therefore, a given MANO does not know if the VNFs it is deploying is a load balancer, firewall, or gateway. Meanwhile, the \gls{do} just coordinates and manages the orchestration process at a given domain, connecting the involved elements such as network systems, SDN controllers, management software, and IT software platforms.

\begin{figure}[t]
    \centering
    \includegraphics[scale=.4]{Figures/02_Background/OrchManaAut}
      \caption{Relationship among orchestration, management, and automation. Both orchestration and management use automation in their processes.}
      \label{diff}
\end{figure}

The \gls{nso} works at a higher level in the control and management stack with interfaces to the OSS/BSS. During a network service creation, the orchestration process can exceed the domains boundaries being necessary to use resources and/or services of other providers or operators. Such resources comprised of physical and virtual components. Thus, the \gls{nso} is supposed to provide service delivery both within single and/or multi-domain environments (more details in Section~\ref{sec:domain}). In this sense, different organizations and telecommunication enterprises have developed many open source projects, driving orchestration evolution towards open standards that it will permit the implementation of products with a large scale of integration. Section~\ref{sec:proj} addresses some of these projects.

In addition, the customers are demanding full information regarding a given hired network service such as detailed pricing, real-time analytics, and a certain control over the service. NSO can offer more information to the customers and put more control into their hand. Its objective is to understand the service profoundly and to enable that providers/operators attend customer demands. 

From an operator and service provider viewpoint, NSO enables to set up new end-to-end services in minutes, keeping those services working and ensuring acceptable performance levels. This process reduces OPEX and provides enhanced services creating new market opportunities and raising the revenues.  As well as, it opens up chances for different companies become service providers or provide virtual network functions.

\gls{nso} is in charge of all network service lifecycle and delivers an end-to-end connectivity service. To achieve so, orchestration is supported by advances in cloud computing, and technologies such as \gls{sdn} and \gls{nfv}, which offer the ability to reconfigure the network quickly as well as programming the forwarding and processing of the traffic. Figure~\ref{nso_rel} aims at illustrating the relationships between \gls{nso}, \gls{nfv}, \gls{sdn}, and Cloud Computing.

\begin{figure}[t]
  \centering
  \includegraphics[scale=.4]{Figures/03_NSO/nso_rel}
    \caption{Illustration of relationships between \gls{nso}, \gls{nfv}, \gls{sdn}, and Cloud.}
    \label{nso_rel}
\end{figure}

Each one of these paradigms has different functions: high level orchestration for \gls{nso}, function programming for \gls{nfv}, networking programming for \gls{sdn},  and resource virtualization for cloud computing. They can work in an integrated pattern to offer advantages such as agility, cost reduction, automation, softwarization and end-to-end connectivity, to enable novel services and applications such as 5G networks.

After this analysis, we can identify the main NSO features as follows: 
\begin{itemize}
\item \textit{High-level vision of the \gls{ns}} that permit an overview of all involved domains, technological and administrative. 
\item \textit{Smart services deployment and provisioning}. These are related to in-deep knowledge about the services, what enable better make decisions. 
\item \textit{Single and multi-domain environment support} that provide deployment of end-to-end service independently of geographical location.
\item \textit{Proper interaction with different MANO and non-MANO elements} which leads to better executed workflows.
\item \textit{New markets opportunities}, offering enhanced services and reducing OPEX.    
\end{itemize}

\subsection{Single and Multi-Domain Orchestration}
\label{sec:domain}

Orchestration in the single and multi-domain environment is different. In a single domain, the orchestrator is in charge of all services and resource availability within its domain as well as has total control over those resources. A domain orchestrator manages the network service lifecycle and interacts with other components not only to control \glspl{vnf}, but also computing, storage, and networking resources. Its scope is limited by administrative boundaries of the provider. As shown in Figure~\ref{mdo}, domain orchestrators can orchestrate heterogeneous technological domains such as \gls{sdn}, \gls{nfv}, Legacy, and Data center. Under single domain environment, it is noticeable that the domain orchestrator works as described by \gls{etsi} in \cite{ETSIIndustrySpecificationGroupISGNFV2014NetworkOptions}. 

However, in a multi-domain environment,  local orchestrators do not know the resources and topologies used by other providers. So, multi-domain orchestration is more complex, since it is supposed to provide end-to-end services, which requires cross-domain information exchange features (cf.~\cite{md2}).  
Currently, there is not a standard for information exchange process in multi-domain environments, either multi-technology domains or multiple administrative domains. There are some multi-domain orchestration candidates, e.g., T-NOVA FP7 project \cite{FP7projectT-NOVAT-NOVAInfrastructures}, ONAP~\cite{onap}, Escape \cite{Sonkoly2015Multi-DomainClouds}, and \gls{5gex} \cite{Bernardos20155GInfrastructures}. All of them will be discussed later in this survey.

\gls{etsi} proposes some options regarding multi-domain orchestration. Initially, \gls{etsi} \gls{nfv}  Release 2 presents two architectures to address multi-domain scenarios~\cite{ETSIIndustrySpecificationGroupISGNFV2014NetworkOptions}. In the first, the \gls{nfvo} is split into \textit{Network Service Orchestrator}, manages the network service, and \textit{Resource Orchestrator}, provides an abstract resource present in the administrative domain. A use case for this first option is illustrated in Figure~\ref{fig:use1}. A Network Operator offers resources to different departments within the same operator, likewise to a different network operator. One or more Data centers and \glspl{vim} represent an administrative domain and provide an abstracted view of its resources (virtual and physical). The Service Orchestrator and VNF Manager can or can not be part of another domain. In this use case, a service can run on the infrastructure provided and managed by another Service Provider.

The second architecture does not split the \gls{nfvo}, but creates a new reference point between NFVOs (See Figure~\ref{fig:use2}) called  Umbrella \gls{nfvo}. This use case requires the composition of services towards deploying a high-level network service. Such service can include network services hosted and offered by different administrative domains. Each domain is responsible for orchestrating its resources and network services. This approach has objectives similar to first, however, an administrative domain is also composed of \glspl{vnfm} (together with their related \glspl{vnf}) and \gls{nfvo}. The \gls{nfvo} provides standard \gls{nfvo} functionalities, with a scope limited to the network services, \glspl{vnf} and resources that are part of its domain.

More recently, the \gls{etsi} \gls{nfv} Release 3 presented others options to support network services across multiple administrative domains~\cite{ETSIGRDomains}. In particular, the use case entitled ``Network Services provided using multiple administrative domains" proposes a multi-domain architecture using \gls{nfvmano}. Such architecture introduces the new reference point named ``Or-Or" between \glspl{nfvo} to enable communication and interoperability. Differently of second option (Figure~\ref{fig:use2}), in this approach, there is a hierarchy between the domains. In the example shown in Figure~\ref{fig:use3}, \gls{nfvo} in Administrative Domain C is on-top, using network services offered by Administrative Domains A and B, as well as managing composite \gls{ns} lifecycle.    

\begin{figure*}[t!]
\centering
 \subfigure[]{
   \includegraphics[scale=.31]{Figures/03_NSO/mdo_etsi_1}
   \label{fig:use1}
 }
 \subfigure[]{
   \includegraphics[scale=.31]{Figures/03_NSO/mdo_etsi_2} 
   \label{fig:use2}
 }
 \subfigure[]{
   \includegraphics[scale=.31]{Figures/03_NSO/mdo_etsi_3} 
   \label{fig:use3}
 }
 \caption{\gls{etsi} approaches for multiple administrative domains: (a) approach in which the orchestrator is split into two components (NSO and RO), (b) approach with multiple orchestrators and a new reference point: Umbrella NFVO, (c) approach that introduces hierarchy and the new reference point Or-Or. Adapted from~\cite{ETSIIndustrySpecificationGroupISGNFV2014NetworkOptions} and~\cite{ETSIGRDomains}.}
 \label{fig:k-clique}  
\end{figure*}

In the scope of this paper, end-to-end network services are composed of one or more network functions interconnected by forwarding graphs. Such services might span multiple clouds and geographical locations. Given that, they require complex workflow management, coordination, and synchronization between multiple involved domains (infrastructure entities), which are performed by one (or more) orchestrator(s). Examples of end-to-end services are virtual extensible LAN (VxLAN), video service delivery, and virtual private network.

\subsection{Orchestrator Functions}
\label{subsec:func}

Section~\ref{sec:def} identifies the various areas of term orchestration. Orchestration can be inserted in the context of cloud, \gls{nfv}, management systems, web services and more recently in the deployment of end-to-end network service in large networks with multiple technologies and administrative domains. In this scope, the orchestrator is the component responsible for automatic resource coordination and control, as well as service provision to customers. 

In the \gls{nfv} context, \gls{etsi} \gls{nfvmano} defines the orchestrator with two main functions including \textit{resources orchestration across multiple \glspl{vim}} and \textit{network service orchestration}~\cite{GSNFV-MAN001:2014}. Network service orchestration functions provided by the \gls{nfvo} are listed below.
\begin{itemize}
\item Management of Network Services templates and \gls{vnf} Packages. This includes validation of templates and packages with the objective of verifying the artifacts' authenticity and integrity. Besides, the software images are cataloged in involved \glspl{pop} using the support of \gls{vim}.
\item Network Service instantiation and management;
\item Management of the instantiation of \glspl{vnfm} and \glspl{vnf} (with support of \glspl{vnfm});
\item Validation and authorization of \gls{nfvi} resource requests from \gls{vnf} managers (resources that impact NS);
\item Management of network service instances topology;
\item Policy management related to affinity, scaling (auto or manual), fault tolerance, performance, and topology.
\end{itemize}

\gls{etsi} \gls{nfvo} functions regarding Resource Orchestration is expressed as follows:
\begin{itemize}
\item Validation and authorization of \gls{nfvi} resource requests from VNF Managers;
\item \gls{nfvi} resource management including compute, storage and network;
\item Collect usage information of \gls{nfvi} resources;
\end{itemize}

Related to \gls{nso}, the orchestrator, in turn, has a  more comprehensive function: to decouple the high-level service layer (e.g., marketplace, network slicing) from underlying management and resources layer (e.g., VNFs, Controller, EMS), simplifying innovations and enhancing flexibility in both contexts. The orchestrator allows complex functions to be implemented in underlying technologies and infrastructures. For example, real-time analytics of network services can be deployed through the orchestrator. Another example, the orchestrator can connect the traditional OSS/BSS to the virtualized infrastructure. The Figure~\ref{orch} represents the significance of the orchestrator in the context of network service.  

The orchestrator creates an abstraction unified point, enabling to abstract physical and virtual resources, transparently exposing them to service providers and other actors, including marketplace and other orchestrators. It gives service providers further control of their network services and enables developers to create new services and functions.

To accomplish this, the orchestrator must be inserted in each layer of telecommunication network stack, from the application layer down to the data plane. Therefore, different orchestrators can exist in each plane, not being limited to a single orchestrator~\cite{Alvizu2016AdvanceEra}. Some of the existing orchestration solutions use an orchestrator logically centralized and consider only ``softwarized'' networks (see Section~\ref{sec:proj}). However, this is impracticable for large and heterogeneous networks. 

The orchestrator can be classified according to its function in: Service Orchestration (SO), Resource Orchestration (RO), and Lifecycle Orchestration (LO). Figure~\ref{funOrch} illustrates the three primary network service orchestrator functions.

The Service Orchestration is responsible for service composition and decomposition. It can be taken as the upper layer, focused in the interaction with other components such as Marketplace and \gls{oss} / \gls{bss}. The Lifecycle Orchestration deals with the management of workflows, processes, and dependencies across service components. Besides, it maintains the services running according to the contracted Service Level Agreement. Finally, the Resource Orchestration is in charge of mapping service requests to resources, either virtual and/or physical. This mapping occurs across elements such as \gls{nfvo}, \gls{ems}, and \gls{sdn} controllers.

\begin{figure}[t!]
  \centering
  \includegraphics[scale=.2]{Figures/03_NSO/orchestrator}
    \caption{Strategic role of the Orchestrator as the glue between the actual services and the underlying management of resources.}
    \label{orch}
\end{figure}

\begin{figure}[t!]
  \centering
  \includegraphics[scale=.35]{Figures/03_NSO/fun_orch}
    \caption{Different orchestrator functions: Resource Orchestration, Service Orchestration, and Lifecycle Orchestration. There is a relationship of dependency and continuity between the functions.}
    \label{funOrch}
\end{figure}

Lifecycle is used to manage a network service with various states (created, provisioned, stopped, etc.). When some action is applied to a network service (e.g., provision a network service), many activities may be needed to apply on the components of this network service. Hence, a workflow is used to execute a bunch of tasks in correct order. Each state of lifecycle can generate one or more activities on workflows. The Figure~\ref{fig:lifeWork} depicts the relationship between lifecycle and workflow of a Network Service. 

Figure~\ref{fig:lifeWork2} presents an example to improve the real definition of lifecycle and workflow in the context of network service. One of the states in the service lifecycle is the \textit{Created}. In order to achieve such state is necessary to execute four tasks: create \gls{vdu}1, create \gls{vdu}2, configure network and run the application. Therefore, the state only is finished when all those activities are completed.

\begin{figure*}[thpb]
\centering
 \subfigure[]{
   \includegraphics[scale=.3]{Figures/03_NSO/life_work}
   \label{fig:lifeWork}
 }
 \subfigure[]{
   \includegraphics[scale=.3]{Figures/03_NSO/life_work2} 
   \label{fig:lifeWork2}
 }    
 \caption{Difference between Lifecycle and Workflow: (a) Lifecycle -- sequence of states and workflow -- activities in correct order and (b) example of network service lifecycle.}
 \label{fig:lifeworkflow}  
\end{figure*}

Service lifecycle automation will allow that requested service remains in a desired state of behavior during its lifetime. With the automation, the system responds proactively to changes network and service conditions without human intervention, getting resilience and faults tolerance. These functional aspects of an orchestrator to guarantee the state of a network towards a service goal are also being referred to as Intent-based Networking (IBN), cf.~\cite{ibn}.

\subsection{Taxonomy}

While many aspects of orchestration are under active development and commercial roll-outs, others are still in a preliminary maturity phase. This subsection enumerates  central concepts and characteristics related to any NSO approach. It becomes very challenging trying to summarize  all concepts related to orchestration in a single work, a challenge exacerbated by  the fast evolving pace of so many moving pieces, from standards to enabling technologies. Figure~\ref{tax} presents the proposed taxonomy as the result of extensive literature research as well as practical  experiences with a number of orchestration platforms and research projects.   

\begin{figure*}[thpb]
  \centering
  \includegraphics[scale=.37]{Figures/03_NSO/taxonomy}
    \caption{\gls{nso} Taxonomy with seven approach: Service Model, Software, Resource, Technology, Scope of Application, Architecture, and Standards \acrfull{sdo}.}
    \label{tax}
\end{figure*}

We identify seven key aspects to characterize network service orchestration: 
\begin{enumerate}
\item \textit{\textbf{Service} Models}. Relates to the type of services unlocked by the \gls{nso}, which may offer new business and relationships and opportunities  (e.g., \gls{vnfaas}, \gls{slaas}).
\item \textbf{\textit{Software}}: Identifies major software-related characteristics of the orchestration solutions, including specificities of the  management and standard interfaces.
\item \textbf{\textit{Resource}}: Refers to the type of underlying resources (e.g., network, compute, and storage) used for the network service deployment.
\item \textbf{\textit{Technology}}: Points to target technologies for \gls{nso} (e.g., Cloud, \gls{sdn}, \gls{nfv}, and Legacy).
\item \textit{\textbf{Scope}}: Considers the application domain in terms of network segments embraced by the network service orchestration (i.e., from access network to data centers).  
\item \textbf{\textit{Architecture}}: Unfolds into three relevant architectural dimensions with relate to single- and multi-domain orchestration and functional  organization.
\item \textit{\textbf{SDO} (Standards Development Organization)}: Relates to standardization activities in scope of the NSO.
\end{enumerate}

Additional sub-areas contribute to an in-depth analysis in different contexts, which are  further  discussed in the following sections. 

\subsubsection{Service Models}
This aspect corresponds to the different service models related to orchestration process. Each service is inserted in the context of cloud, \gls{sdn}, and/or \gls{nfv}. Cloud computing offers three categories of services such as \gls{iaas}, \gls{paas} and \gls{saas} \cite{Leavitt2009}. In \gls{iaas}, Cloud Service Provider (CSP) renders a virtual infrastructure to the customers. In \gls{paas}, CSPs provide development environment as a service. Finally, \gls{saas} is a service that furnishes applications hosted and managed in the cloud. 

\gls{sdn} and \gls{naas} paradigms can be gathered to provide end-to-end service provisioning. While SDN supply the orchestration of underlying network (switches, router, and links), the \gls{naas} is responsible for private access to the network and customer security~\cite{Karakus2017QualitySurvey}. 

The \gls{nfv}, in turn, can offer new services including \gls{nfviaas}, \gls{vnfaas}, \gls{slaas} and \gls{vnpaas}. The \gls{nfviaas} provides jointly \gls{iaas} and \gls{naas} tailored for \gls{nfv}. \gls{vnfaas} is a service that implements virtualized Network Functions to the Enterprises and/or end customers. \gls{vnpaas} is a platform available by service providers allowing customers to create their own network services. The \gls{slaas} is a concept that the slices are traded and used to build infrastructure services.

All these services can work in parallel to offer higher-level services. Each one acts in a specific area and offers features to customers, enterprises, or other providers.   

\subsubsection{Software} 
There are many software artifacts related to orchestration covering from a single cloud environment up to more complex scenarios involving multi-domain orchestration. These solutions are outcomes of open source initiatives, research projects or commercial vendors.   

Open source approaches significantly accelerate consensus, delivering high performance, peer-reviewed code that forms a basis for an ecosystem of solutions. Open source makes it possible to create a single unified orchestration abstraction.  Thus, both research projects and commercial vendors leverage open source technologies to accelerate and improve their solutions. Operators, such as Telefonica, China Mobile, AT\&T, and NTT, appear committed to using open source as a way to speed up their development of orchestration platforms~\cite{Sdxcentral20162016:}.

The \gls{osi}\footnote{http://opensource.org} defines licenses under Open Source Definition compliance, which allows code and software to be freely used, shared and modified. The more popular open source licenses are Apache License 2.0, \gls{bsd}, GNU General Public License (GPL), Mozilla Public License 2.0, and Eclipse Public License. Namely, the most important orchestration projects and frameworks (for instance, Aria, Cloudify, CORD, Gohan, Open Baton, Tacker, ONAP, SONATA, and T-NOVA) present a widespread usage of Apache License 2.0.

Another topic related to open source is governance. In short, governance defines the processes, structures, and organizations. It determines how power is exercised and distributed and how decisions are taken. Commonly, a governing board is responsible for the budget, trademark/legal, marketing, compliance, and overall direction, while a technical steering committee is responsible for technical guidance. 

An open source orchestration project may be organized as a single community (e.g. vendor-lead) or can be hosted (and eventually integrated with other projects) by a foundation entity~\cite{Opensource.comFourOpensource.com}. A remarkable example is the Linux Foundation, which among multiple networking related projects is in charge of  ONAP, an open source platform aiming at the  automation, design, orchestration, and management of SDN and NFV services. Another noteworthy example of an orchestration open source project under the Linux Foundation flagship  aiming at delivering a standard \gls{nfv}/\gls{sdn} platform for the industry is Open Platform for NFV (OPNFV)~\cite{LinuxFoundation}.

NSO solutions need to perform management tasks such as remote device configuration, monitoring and fault management. Moreover, they require defining interface of communication between various software components. For this, there are diverse types of management and  standard interfaces such as \gls{cli}, \gls{api}, and \gls{gui}. The \gls{cli} just is used to execute commands directly in the software using remote access via SSH or Telnet. The \gls{api} enables the remote management and interconnection with other softwares through specifics commands. The majority of solutions use REST-based \gls{api}. \gls{gui}, in turn, offers a graphic interface that makes it easier its use.   

\subsubsection{Resource}
During the creation of a network service, the resource orchestration is responsible for orchestrating the underlying infrastructure. Such infrastructure is composed of heterogeneous hardware and software for hosting and connecting the network services. The resources include compute, storage, and network~\cite{Ordonez-Lucena2017NetworkChallenges}. In essence, there are three types of networks: packet, optical and wireless (e.g., Wi-Fi, wi-max, and mobile network).

Resources are shared and abstracted making use of virtualization techniques (e.g., para-virtualization~\cite{4299349}, full virtualization~\cite{4482796}, and containers~\cite{6906035}), defining virtual infrastructures that can be used as physical ones. Sharing and management of resources are much more complex in multi-domain scenarios. The NSO needs to know all the available resources towards the efficient deployment of the \glspl{ns}.  

\subsubsection{Technology}
NSO involves complex workflows and different technologies involved throughout orchestration process: cloud computing, \gls{sdn}, \gls{nfv}, and legacy.      

The cloud computing paradigm provides resource virtualization and improves resource availability and usage by means of orchestration and management procedures. This includes automatic instantiation, migration and snapshot of \glspl{vm}, High-Availability, and dynamic allocation of resources~\cite{ETSI2012NetworkAction}. 

The \gls{sdn} promotes control across network layers and logical centralization of network infrastructure management. Its main functions is to connect the \glspl{vnf} and the \gls{nfvi}-\glspl{pop}. In parallel, the \gls{nfv} technology promotes the network functions programming in order to enable elasticity, automation, and resilience in cloud environments \cite{Rotsos2017NetworkSurvey}. As illustrated in Figure~\ref{nso_rel}, cloud computing, \gls{sdn} and \gls{nfv} are enabler technologies to the \gls{nso}. The NSO must also handle legacy technologies such as MPLS, BGP, SONET / SDH, and WDM. 

\subsubsection{Scope}

Resources of operators under an orchestration application domain can be part of access networks, aggregation networks, core networks and data centers~\cite{5GPPPArchitectureWorkingGroup2016ViewArchitecture}. The access network is the entry point which connects customers to their service provider. It encompasses various technologies, i.e., fixed access, wireless access (Wi-Fi, LTE, radio, WiMAX), optical, and provide connectivity to heterogeneous services such as mobile network and \gls{iot}. The core network is the central part of a telecommunications network that connects local providers to each other. The aggregation network, in turn, connects the access network to core network. The data center is the local where are localized the computing and storage resources.

The infrastructure is formed by heterogeneous technologies that may be owned by different infrastructure providers. The network service orchestration in this environment is a challenging task. The \gls{nso} must have a view of resources and services since access network until the data center to deploy end-to-end network services. Besides, it is also important to provide consistent and continuous service, independent of the underlying infrastructure~\cite{5GPPPArchitectureWorkingGroup2016ViewArchitecture}. 

\subsubsection{Architecture}
An NSO architecture can be divided into three sub-categories: (i) \textit{domain}, (ii) \textit{organization}, and (iii) \textit{functions}. The \textit{domain} refers to coverage of the orchestration process in one or more administrative domains: single-domain and multi-domain. In each scenario, the orchestration has its peculiarities and challenges.

Single-domain orchestration studies focus on vertical \gls{nfv}/\gls{sdn} orchestration within a same administrative domain. In our definition, an administrative domain can have multiple technological domains, such as \gls{sdn}, \gls{nfv}, and Legacy. The taxonomy is aligned with \gls{etsi} \gls{nfv} architecture that addresses orchestration for \gls{nfv}. The multi-domain orchestration involves the instantiation of network service among two or more administrative domains. It is composed of planes (or layers) with different functions and architecture topology. The multi-domain interfaces are not present in original \gls{etsi} \gls{nfv} architecture

The \textit{organization} refers to the different architectural arrangements of a \gls{nso} solution. We identified three types of organization: hierarchical, cascading and distributed. The hierarchical approach assumes a high-level orchestrator that has visibility of the entire other domains and capable of configuring services across different domains. The service provider facing the customer as a single entry point will maintain relationships with other providers to complete the requested service. According to \cite{Bohn2011NISTArchitecture}, the hierarchical approach is impractical because of scalability and trust constraints.  
Under the cascade model, the provider partially satisfies the service request but complements the service by using resources from another provider. If this provider does not have all the resources, it also can request for another and so on (e.g., a mobile network provider using a satellite provider). In the distributed model, there is not a central actor, and providers request resource and services from each other on a peer-to-peer fashion.

Finally,  \textit{functions},as discussed in Sec.~\ref{subsec:func}, refers to the main tasks developed by network service orchestrator: service orchestration, resource orchestration, and lifecycle orchestration. These functions can be separated or together in the same component of an orchestration framework. This decision depends on how the orchestrator was developed.

\subsubsection{Standards Development Organization (SDO)}
Several Standards Development Organizations, including \gls{etsi}, \gls{mef}, \gls{ietf}, and \gls{itu}, are actively working on a collection of standards in order to define reference architectures, protocols, and interfaces in scope of the orchestration domain. Besides, other organizations, academic, vendors and industrial are working in parallel with diverse goals. The main efforts within standardization bodies will be outlined next.
\input{TEXs/04_NSO_Standardization}
\input{TEXs/05_Projects}
\section{Enabling Technologies and Solutions}
\label{sec:proj}
Some of the existing orchestrating solutions are just tied to a specific networking environment, and moreover, some of them can orchestrate an only limited number of services~\cite{Kuklinski2016DesignOrchestrators}. In this section, an overview of main orchestration frameworks is presented, including open source, proposed and commercial solutions. The projects cover different technologies and domains. The Table~\ref{tab:NSOsolutions} summarizes the main characteristics of each open source projects with respect to leader entities, resource domains, scope \gls{nfvmano}, \gls{vnf} definition, Management Interface, and coverage area (single/multi-domain).

\subsection{Open Source Solutions}

Open Source Foundations such as the Apache Foundation and the Linux Foundation are increasingly becoming the hosting entities for large collaborative open source projects in the area of networking.  
The most important projects are \gls{onos}, \gls{cord}, Open Daylight, OPNFV and, recently, ONAP, formed by the merger of \gls{openo} and ECOMP. All the projects are important to create a well-defined platform for service orchestration.

Note that to 5G network, standardization and open source are essential for fast innovation. Vendors, operators, and communities are betting on open source solutions. Even so, existing solutions are still not mature enough, and advanced network service orchestration platforms are missing~\cite{Katsalis2016Multi-DomainDirections}.

In early 2016, the Linux Foundation formed the \gls{openo} Project to develop the first open source software framework and orchestrator for agile operations of \gls{sdn} and \gls{nfv}. \gls{onos} is also developing an orchestration platform for the \gls{cord} project to provide \gls{xaas} exploiting \gls{sdn}, micro-services and disaggregation using open source software and commodity hardware~\cite{Alvizu2016AdvanceEra}.

Many open source initiatives towards network service orchestration are being deployed and this including operators, \gls{vnf} vendors and integrators. However, these are still in the early stages. We describe next some of these initiatives.

\subsubsection{Open Source MANO}
\gls{etsi} Open Source MANO~\cite{ETSIOpenMANO} is an ETSI-hosted project to develop an Open Source \gls{nfvmano} platform aligned with \gls{etsi} \gls{nfv} Information Models and that meets the requirements of production \gls{nfv} networks. The project launched its third release~\cite{Israel2017OSMOverviewb} in October 2017 and presented improvements in security, service assurance, resilience, and Interoperability. One of the main goals of this project is to promote the integration between standardization and open source initiatives.

The \gls{osm} architecture has a clear split of orchestration function between Resource Orchestrator and Service Orchestrator. It integrates open source software initiatives such as Riftware as Network Service Orchestrator and GUI, OpenMANO as Resource Orchestrator (\gls{nfvo}), and Juju~\footnote{https://www.ubuntu.com/cloud/juju} Server as Configuration Manager (G-VNFM). The resource orchestrator supports both cloud and SDN environments. The service orchestrator provides \gls{vnf} and NS lifecycle management and consumes open Information/Data Models, such as YANG. Its architecture covers only single administrative domain.  

\subsubsection{Tacker}
Tacker~\cite{OpenStackFoundation2016} is an official OpenStack project building a Generic \gls{vnfm} and a \gls{nfvo} to deploy and operate Network Services and \glspl{vnf} on a Cloud/\gls{nfv} infrastructure platform such as OpenStack. It is based on \gls{etsi} \gls{mano} architectural framework and provides a functional stack to orchestrate end-to-end network services using \glspl{vnf}.

The \gls{nfvo} is responsible for the high-level management of \glspl{vnf} and managing resources in the \gls{vim}. The \gls{vnfm} manages components that belongs to the same \gls{vnf} instance controlling the \gls{vnf} lifecycle. The Tacker also does mapping to SFC (Service Function Chain) and supports auto scaling and TOSCA \gls{nfv} Profile (using heat-translator).

The tacker components are directly integrated into OpenStack and thus provides limited interoperability with others \glspl{vim}. It combines the \gls{nfvo} and \gls{vnfm} into a single element nevertheless, internally, the functionalities are divided. Another limitation is that it just works in single domain environments.   

\subsubsection{Cloudify}
Cloudify~\cite{GigaSpaces2015} is an orchestration-centric framework for cloud orchestration focusing on optimization \gls{nfv} orchestration and management. It provides a \gls{nfvo} and Generic-\gls{vnfm} in the context of the \gls{etsi} \gls{nfv}, and can interact with different \glspl{vim}, containers, and non-virtualized devices and infrastructures. Cloudify is aligned with the \gls{mano} reference architecture but not fully compliant. 

Besides, Cloudify provides full end-to-end lifecycle of \gls{nfv} orchestration through a simple TOSCA-based blueprint following a model-driven and application-centric approach. It includes \gls{aria} as its core orchestration engine providing advanced management and ongoing automation.

In order to help contribute to open source \gls{nfvmano} adoption, Cloudify engages in and sponsors diverse \gls{nfv} projects and standards organizations, such as TOSCA specification, \gls{aria} and \gls{onap}.

\subsubsection{ONAP}
Under the Linux Foundation banner, \acrfull{onap}~\cite{onap} resulted from the union of two open source \gls{mano} initiatives (OPEN-O~\cite{Foundation} and OpenECOMP~\cite{ATT2016ECOMPPaper}). The \gls{onap} software platform deploys a unified architecture and implementation, with robust capabilities for the design, creation, orchestration, monitoring and lifecycle management of physical and virtual network functions~\cite{onapwiki}. Also, the \gls{onap} functionalities are expected to address automated deployment and management and policies optimization through an intelligent operation of network resource using big data and \gls{ai}~\cite{onapconvergedigest}.

Two of the biggest challenges to merge two large sets of code are: (i) define a higher-level common information model unifying the predominant data models used by OPEN-O (TOSCA) and OpenECOMP (YANG) and, (ii) create a standard process to the onboarding and lifecycle management of VNFs so that end users can introduce these using an automated process (without requiring core developer teams)~\cite{onaplightreading}.

\subsubsection{X--MANO}
X--MANO~\cite{francescon2017x} is an orchestration framework to coordinate end-to-end network service delivery across different administrative domains. X--MANO introduces components and interfaces to address several challenges and requirements for cross-domain network service orchestration such as (i) business aspects and architectural considerations, (ii) confidentiality, and (iii) life-cycle management. In the former case,  X--MANO supports hierarchical, cascading and peer-to-peer architectural solutions by introducing a flexible, deployment-agnostic federation interface between different administrative and technological domains. The confidentiality requirement is addressed by the introduction of a set of abstractions (backed by a consistent information model) so that each domain advertises capabilities, resources, and \glspl{vnf} without exposing details of implementation to external entities. To address the multi-domain life-cycle management requirement, X--MANO introduces the concept of programmable network service based on a domain specific scripting language to allow network service developers to use a flexible programmable Multi-Domain Network Service Descriptor (MDNS), so that network services are deployed and managed in a customized way.

\subsubsection{Open Baton}
Built by the Fraunhofer Fokus Institute and the Technical University of Berlin, Open Baton~\cite{openbatongit} is an open source reference implementation of the NFVO based on the ETSI NFV MANO specification and the TOSCA Standard. It allows it to be a vendor-independent platform (i.e., interoperable with different vendor solutions) and easily extensible (at every level) for supporting new functionalities and existing platforms.

The current Open Baton release 3 includes many different features and components for building a complete environment fully compliant with the NFV specification. Among the most important are: (i) a \gls{nfvo} (exposing TOSCA APIs) , (ii) a generic \gls{vnfm} and Juju \gls{vnfm}, (iii) a marketplace integrated within the Open Baton dashboard, (iv) an Autoscaling and Fault Management System and (v) a powerful event engine for the dispatching of lifecycle events execution.

Finally, Open Baton is included as a supporting project in the project named Orchestra\footnote{https://wiki.opnfv.org/display/PROJ/Orchestra}. This OPNFV initiative seeks to integrate the Open Baton orchestration functionalities with existing OPNFV projects in order to execute testing scenarios (and provide feedbacks) without requiring any modifications in their projects.

\subsubsection{ARIA TOSCA}
Under the Apache Software Foundation, Agile Reference Implementation of Automation~(ARIA)~\cite{ariatosca} is a framework for building TOSCA-based orchestration solutions. It supports multi-cloud and multi-VIM environments while offering a Command Line Interface~(CLI) to develop and execute TOSCA templates, and an easily consumable Software Development Kit~(SDK) for building TOSCA enabled software. By taking advantage of its programmable interface libraries, ARIA can be embedded into collaborative projects that want to implement TOSCA-based orchestration. For example, Open-O~\cite{Foundation} is using the ARIA TOSCA code-base to create its SDN \& NFV orchestration tool~\cite{ariatoscacloudify}.

\subsubsection{XOS}
Designed around the idea of Everything-as-a-Service (XaaS), XOS~\cite{peterson2015xos} unifies SDN, NFV, and Cloud services (all running on commodity servers) under a single uniform programming environment. The XOS software structures is organized around three layers: (i) a Data Model (implemented in Django\footnote{https://www.djangoproject.com/}) which records the logically centralized state of the system, (ii) a set of Views (running on top of the Data Model) for customizing access to the XOS services and (iii) a Controller Framework (from-scratch program) is responsible for distributed state management. 

XOS runs on the top of a mix of service controllers such as data center cloud management systems (e.g., OpenStack), SDN-based network controllers (e.g., ONOS), network hypervisors (e.g., OpenVirtex), virtualized access services (e.g., CORD), etc. This collection of services controllers allows the mapping to XOS onto the ETSI NFV Architecture playing the role of a \gls{vnfm}. Using XOS as the \gls{vnfm} facilitates unbundling the gls{nfvo} and enable to control both a set of EMs and the VIM~\cite{xos}.

\subsubsection{TeNOR}
Developed by the T-NOVA project~\cite{FP7projectT-NOVAT-NOVAInfrastructures}, the main focus of this Multitenant/Multi NFVI-PoP orchestration platform is to manage the entire \gls{ns} lifecycle service, optimizing the networking and IT resources usage. TeNOR~\cite{7502419} presents an architecture based on a collection of loosely coupled, collaborating services (also know as micro-service architecture) that ensure a modular operation of the system. Micro-services are responsible for managing, providing and monitoring \gls{ns}/\glspl{vnf}, in addition to forcing SLA agreements and determining required infrastructure resources to support a NS instance. 

Its architecture is split into two main components: \textit{Network Service Orchestrator}, responsible for NS lifecycle and associated tasks, and \textit{Virtualized Resource Orchestrator}, responsible for the management of the underlying physical resources. To map the best available location in the infrastructure, TeNOR implements service mapping algorithms using \gls{ns} and \gls{vnf} descriptors. Both descriptors follow the TeNOR's data model specifications that are a derived and extended version of the ETSI NSD and VNFD data model.

\subsubsection{Gohan}
NTT's Gohan~\cite{gohan} is a MANO-related initiative for \gls{sdn} and \gls{nfv} orchestration. The Gohan architecture is based on micro-services (just as the TeNOR implementation) within a single unified process in order to keep the system architecture and deployment model simple. A Gohan service definition uses a JSON schema (both definition and configuration of resources). With this schema, Gohan delivers a called schema-driven service deployment, and it includes REST-based API server, database backend, command line interface (CLI), and web user-interface (WebUI). Finally, a couple of applicable use cases for the NTT's Gohan include to use it (i) in the Service Catalog and Orchestration Layer on top of Cloud services and (ii) as a kind of NFV MANO which manages both Cloud VIM and legacy network devices. 

\subsubsection{ESCAPE}
Based on the architecture proposed by EU FP7 UNIFY project~\cite{unify}, ESCAPE (Extensible Service ChAin Prototyping Environment) is a NFV proof of concept framework which supports three main layers of the UNIFY architecture: (i) service layer, (ii) orchestrator layer and, (iii) infrastructure layer~\cite{csoma2014escape}. It can operate as a Multi-domain orchestrator for different technological domains, as well as different administrative domains. ESCAPE also supports remote domain management (recursive orchestration), and it operates on joint resource abstraction models (networks and clouds)~\cite{sonkoly2015multi}.  

In the current implementation of the process flow in ESCAPE, it receives a specific service request on its REST API of the Service Layer. It then sends the requested Service Function Chains to the Orchestration Layer to map the service components to its global resource view. As a final step, the calculated service parts are sent to the corresponding local orchestrators towards instantiating the service.

\begin{table*}[t]
\centering
\rowcolors{2}{gray!25}{}
\renewcommand{\arraystretch}{1.3}
\setlength{\arrayrulewidth}{1pt}
\scriptsize
\caption{Summary of Open Source NSO Implementations}
\label{tab:NSOsolutions}
\begin{tabular}{p{1.2cm}p{1.7cm}p{1.7cm}|c|c|c|c|c|c|c|c|c|c|c|c|}
\multirow{2}{*}{Solution} & \multirow{2}{*}{Leader} & \multirow{2}{*}{VNF Definition} & \multicolumn{4}{c|}{Resource Domain}                                                                           & \multicolumn{3}{c|}{MANO}                                                        & \multicolumn{3}{c|}{Interface Management}                                      & \multicolumn{2}{c|}{Domain}                                 \\
                          &                         &                                 & \multicolumn{1}{l|}{Cloud} & \multicolumn{1}{l|}{SDN} & \multicolumn{1}{l|}{NFV} & \multicolumn{1}{l|}{Legacy} & \multicolumn{1}{l|}{NFVO} & \multicolumn{1}{l|}{VNFM} & \multicolumn{1}{l|}{VIM} & \multicolumn{1}{l|}{CLI} & \multicolumn{1}{l|}{API} & \multicolumn{1}{l|}{GUI} & \multicolumn{1}{l|}{Single} & \multicolumn{1}{l|}{Multiple} \\ \hline\hline 
ARIA TOSCA                & Apache Foundation       & TOSCA                           &  \ding{51}                 &                          &                          &                             &                           &                           &                          &        \ding{51}                   &         \ding{51}                  &                          &          \ding{51}                    &                               \\
Cloudify                  & GigaSpace               & TOSCA                           &    \ding{51}                         &                          &      \ding{51}                    &                             &     \ding{51}                       &      \ding{51}                      &                          &           \ding{51}                &        \ding{51}                   &         \ding{51}                  &      \ding{51}                        &                               \\
ESCAPE                    & FP7 UNIFY               & Unify                           &      \ding{51}                       &       \ding{51}                    &       \ding{51}                   &                             &      \ding{51}                      &                           &       \ding{51}                    &        \ding{51}                   &   \ding{51}                        &                           &           \ding{51}                  &         \ding{51}                      \\
Gohan                     & NTT Data                & Own                             &      \ding{51}                       &      \ding{51}                    &       \ding{51}                   &     \ding{51}                        &       \ding{51}                    &        \ding{51}                   &                          &       \ding{51}                   &       \ding{51}                   &      \ding{51}                    &          \ding{51}                   &                               \\
ONAP                      & Linux Foundation        & HOT, TOSCA, YANG                &       \ding{51}                      &     \ding{51}                     &     \ding{51}                     &    \ding{51}                         &      \ding{51}                     &        \ding{51}                   &    \ding{51}                       &      \ding{51}                    &    \ding{51}                      &      \ding{51}                    &        \ding{51}                     &        \ding{51}                       \\
Open Baton                & Fraunhofer / TU Berlin  & TOSCA, Own                      &        \ding{51}                     &                          &        \ding{51}                  &                             &       \ding{51}                    &       \ding{51}                    &                          &      \ding{51}                    &         \ding{51}                 &         \ding{51}                 &        \ding{51}                     &                               \\
OSM                       & ETSI                    & YANG                            &        \ding{51}                     &       \ding{51}                   &      \ding{51}                    &                             &     \ding{51}                      &          \ding{51}                 &          \ding{51}                &        \ding{51}                  &      \ding{51}                    &           \ding{51}               &        \ding{51}                     &                               \\
Tacker                    & OpenStack Foundation    & HOT, TOSCA                      &        \ding{51}                     &                          &        \ding{51}                  &                             &     \ding{51}                      &       \ding{51}                    &                          &     \ding{51}                     &          \ding{51}                &        \ding{51}                  &     \ding{51}                        &                               \\
TeNOR                     & FP7 T-NOVA              & ETSI                            &         \ding{51}                    &        \ding{51}                  &       \ding{51}                   &                             &        \ding{51}                   &                           &                          &                          &        \ding{51}                  &         \ding{51}                 &        \ding{51}                     &                               \\
X-MANO                    & H2020 VITAL             & TOSCA                           &                            &                          &         \ding{51}                 &                             &    \ding{51}                       &                           &                          &                          &       \ding{51}                   &   \ding{51}                       &                             &          \ding{51}                     \\
XOS                       & ON.Lab                  & \multicolumn{1}{c|}{-}          &         \ding{51}                    &      \ding{51}                    &       \ding{51}                   &                             &                           &        \ding{51}                   &                          &                          &    \ding{51}                      &       \ding{51}                   &          \ding{51}                   &     \ding{51}      \\ \hline                  
\end{tabular}
\end{table*}




\subsection{Commercial Solutions}

The commercial orchestration solutions market is rising and will be formed by diverse types of companies including new startups, service provider IT vendors, VNF vendors, and the traditional network equipment vendors~\cite{Sdxcentral2016LifecycleOverview}.    

Some software and hardware vendors already offer network orchestration solutions. Below are presented the major commercial products that we consider as mature and robust solutions. All information about the products was got through the vendor's site and technical reports.

%The 
Cisco offers a product named Network Services Orchestrator enabled by Tail-f~\cite{CiscoIncNetworkCisco}. It is an orchestration platform that provides lifecycle service automation for hybrid networks (i.e., multi-vendors). Cisco NSO enables to design and deliver services faster and proposes an end-to-end orchestration across multiple domains. The platform deploys some management and orchestration functions such as \gls{nso}, Telco cloud orchestration, \gls{nfvo}, and \gls{vnfm}.    

The Blue Planet SDN/NFV Orchestration platform~\cite{BluePlanet2017BLUESUITE} is a Ciena's solution that provides an integration of orchestration, management and analytics capabilities. It aims to automate and virtualize network service across physical and virtual domains. The platform supports multiple use cases, including SD-WAN service orchestration, NFV-based service automation, and \gls{cord} orchestration.

Another commercial solution is the HPE Service Director of the Hewlett Packard Enterprise. The product is a service orchestration \gls{oss} solution that manages end-to-end service and provides analytics-based planning and closed-loop automation using declarations-based service model. It supports multi-vendor VNF, multi-VIM, various OpenStack flavors and multiple SDN controllers.

The Oracle Communications Network Service Orchestration solution~\cite{OracleCommunicationsOracleSolution} orchestrates, automates, and optimizes VNF and network service lifecycle management by integrating with BSS/OSS, service portals, and orchestrators. It has two environments to deploy the network services: one design-time environment to design, define and program the capabilities, and a run-time execution environment to execute the logic programmed and lifecycle management. In essence, it plays the roles of the \gls{nfvo}, Telco cloud orchestration, and end-to-end service.  

Ericsson offers some solutions in the scope of the cloud, \gls{sdn} and orchestration. One of them is the Ericsson Network Manager~\cite{EricssonInc.EricssonManager} that provides a unified multi-layer, multi-domain (\gls{sdn}, \gls{nfv}, radio, transport and core) management systems and plays various roles such as \gls{vnfm}, network slicing, and network analytics. 

Many of the above-mentioned products  are often extensions of proprietary platforms. There are few details publicly available, mostly marketing material. The list of commercial solutions is not exhaustive and will certainly become outdated. However, the overview should serve as a glimpse on the expected rise of commercial NSO solutions in the near future as  enabling open source technologies and standards mature.
\input{TEXs/07_Enabling_Scenarios}
\section{Challenges and research opportunities}
\label{sec:challenge}

\gls{nso} promises to improve efficiency when instantiating (day~1) and operating (day~2) network services, but the path ahead is not without  challenges. 
This section provides a discussion on the main challenges and research opportunities for \gls{nso}, including scalability, security, resource modeling, performance, and interoperability.

\subsection{Interoperability}

Typically, operators infrastructures are organized in several domains that differ in geographical locations, management (e.g., legacy or \gls{sdn}), administrative boundaries, and technologies. One of the challenges for service providers is to create and to manage services across unique and proprietary interfaces, making integration and startup difficult tasks to be achieved, as well as increasing the operational costs.  

In this scenario, interoperability is essential to enable the deployment of end-to-end network services. Few end-to-end services will be confined within the boundaries of a single domain. They normally encompass a multi-domain orchestration environment composed of providers and vendors with different incentives and business models~\cite{Katsalis2016Multi-DomainDirections}. There is no consensus about how would be the exchanging process between the multiple actors in deployment end-to-end network services. In fact, \gls{etsi} \gls{mano} architecture does not bring any provisioning for this kind of exchange~\cite{ETSIIndustrySpecificationGroupISGNFV2014NetworkNFV}. 

A number of orchestration solutions based on the ETSI MANO architecture have emerged with the objective of proposing a complete orchestration framework. Table~\ref{tab:NSOsolutions} shows notable solutions. Although the progress made by ETSI in defining architecture and interfaces, each solution uses a particular implementation and data model, which makes interoperability difficult to achieve (cf.~\cite{NOn}). As a result, chaining network functions leveraging different solutions for a single network service deployment and operation is currently a very costly proposition regarding development efforts and time-to-market.   

Standardization is a path to enable interoperability of network services between operators and address limitations that arise in the deployment of services, as explained in Section~\ref{sec:stand}. Another parallel track towards interoperability is a broad adoption of software components and broad agreements on APIs along data and information models fueled by re-usable open source artifacts. 

\subsection{Resource and Service Modeling}

Network services need to be efficiently modeled towards deploying resource requirements, configuration parameters, management policies, and performance metrics. Service modeling will enable abstraction of resources and capabilities of underlying layers. It simplifies the understanding of functions and provides a generic way to represent resource and service. 

However, it is a major challenge to translate higher-level policies, which are generated from the resource allocation and optimization mechanisms, into a lower level configuration. Templates and standards should be developed to guarantee automated and consistent translation~\cite{YongLi2015Software-DefinedSurvey}. Besides, the standardization can enable the interoperability and integration of network services templates and addresses limitations arising in the deployment of services in heterogeneous landscape.

There are templates and data modeling languages for \acrfull{nfv} and \acrfull{ns} such as TOSCA, YANG, and HOT. In addition, some organizations propose their approaches to the definition of Network Services, e.g., Open Baton and Gohan.

\gls{etsi} \gls{nfv} \gls{mano} proposes VNF and Network Service descriptors as templates for the definition of functions and services. According to \gls{etsi}, \gls{ns} is defined as a set of \glspl{vnf} and/or \glspl{pnf} interconnected by \glspl{vl} and one or more \acrlong{vnffg}. 

On the other hand, \gls{etsi} \gls{ns} specifies lowest level resources such as CPU, memory, and network, but it does not extend the resource modeling and does not define a data model to the descriptors~\cite{Mijumbi2016ManagementVirtualizationb}. Thus, its approach is driven to single domain environment~\cite{Garay2016ServiceForward}. 

On the other hand, the \gls{ietf} \gls{sfc} provides the ability to define an ordered list of network services, or service functions (e.g., firewalls, load balancers, DPI) connecting them in a virtual chain. However, \gls{sfc} does not describe the underlying resource, since its primary focus is service operation, apart from the forwarding topology. As opposed to ETSI, SFC scope covers multi-domain connections.   

Resource and service modeling in softwarized networks including multi-domain scenarios need further work. This evolution will enable interoperability of network services and the correct mapping between the high-level configuration and the underlying infrastructure. Currently, the interoperability among the diverse orchestration platforms does not exist.

\subsection{Network Service Lifecycle Management}

Network service lifecycle consists in all process for deployment, execution, and termination of a network service. The Network Service Lifecycle Management is fundamental to ensure the correct operation of the service.

Nevertheless, the network services can have specific lifecycle management requirements. For example, an NS can use specific resources as \gls{sriov}~\cite{5416637} and DPDK or need resources across various domains. This type of requirements becomes harder the service deployment.

One possible solution is service lifecycle automation. It enables lifecycle management without human intervention. Automation can be obtained through heuristic algorithms and machine learning techniques. ONAP is working on new closed control loops (e.g., CLAMP - Closed Loop Automation Management Platform)\footnote{https://github.com/onap/clamp} towards providing automation, performance optimization and Service Lifecycle Management, eventually leveraging network analytics and machine learning assisted decisions.
Nevertheless, many aspects of run-time (day 2) workflow modeling and implementation remain open, with TOSCA extensions and BPMN/BPML approaches~\cite{DBLP:conf/closer/CalcaterraCMT17} undergoing improvements to meet the needs of NSO-based lifecycle automation. 


\subsection{Performance and  Service Assurance}

The changes that orchestration technology brings to the telecommunication infrastructures make them increasingly virtualized and software-based. So, performance is a constant challenge in a highly dynamic environment of virtual functions and services.  

This change reflects on enabling technologies. For instance, the \gls{nfv} should meet performance requirements to support, in a standard server, the packet processing, including high I/O speed, fast transmission, and short delays~\cite{YongLi2015Software-DefinedSurvey}. The \glspl{vnf} must achieve a performance comparable to specialized hardware. According to~\cite{Mijumbi2016NetworkChallenges}, some applications require specific capabilities, but virtualization can degrade their performance. This generates a trade-off between performance and flexibility. However, recent advances in CPU and virtualization technologies are overcoming these challenges include \gls{dpdk}~\cite{LinuxFoundationDPDKKit} -- libraries and drivers for fast packet processing, NetVM~\cite{7036139} -- enabling high bandwidth network functions to operate at near line speed, and ClickOS~\cite{Martins:2014:CAN:2616448.2616491} -- minimalist operating that supports high throughput, low delay, and isolation. Likewise, the document~\cite{ETSIGSPractises} of the \gls{etsi}  provides a set of recommendations on the minimum requirements that the hardware and virtualized layer should have to achieve high performance.

Another question is performance monitoring coupled with Network Services maintenance. Both require a global view of the resources and a unified control and optimization process with various optimization policies running in it. The monitoring is required to avoid the violation of \glspl{sla} in the Service layer. In order to keep NS performance, it is demanded that the system equally performs in different layers. In multi-domain scenarios, this becomes more complex because it is necessary the exchange of information and resources between different organizations/domains~\cite{md2}. 
VNF benchmarking~\cite{7313620} and NS chain profiling~\cite{7956044} coupled to NSO lifecycles and run-time MANO resource allocation and management decisions are potential techniques towards service guarantees and SLA compliance.  

In addition, a better composition between the traffic forwarding and \gls{nf} placement is required towards optimizing the \gls{ns} deployment. The first steps to provide service performance guarantees are to avoid heavily loaded service nodes and to identify bottleneck links. Algorithms and machine learning techniques can archive better results in this composition.   

Thus, how to achieve high performance is an important problem in the research and development of \gls{nso} solutions. Projects within the 5G Infrastructure Public Private Partnership (5G-PPP)~\cite{elayoubi:hal-01488208} are targeting enhanced performance towards better user experience. 

\subsection{Scalability}

Some studies assume that 5G network might connect 50 billion devices until 2020~\cite{Panwar2016ACommunication},~\cite{Evans2011TheEverything}. This growth is due to the emergence of vertical industries such as Internet of Things, Smart Cities, and Sensor Networks. In this scenario, orchestration process requires the ability to handle the growth of networks and services to support the huge amount of connected nodes.

In addition, the network services can be deployed over different domains managed by third parties, infrastructure covering large geographical space and diverse type of resources such as access, transport, and core networks. This environment demands high scalability of the components involved, including orchestrators, controllers, and managers. 

Most current \gls{nso} use cases are just based on deploying a network service in a controlled scenario. Just a use case is not able to check the scalability of the solution. In a production environment, the orchestrator is responsible for orchestrating millions of customers and services at the same time. Hence, scalability is an important feature for \gls{nso} success.

Some orchestration solutions mainly focus on centralized solutions, which pose scalability issues. The works~\cite{Alvizu2016AdvanceEra} and~\cite{Garay2016ServiceForward} suggest different orchestrators involved in the orchestration process of end-to-end network services, not being limited to a single orchestrator. However, there are several particularities on each layer that could be better explored with specific orchestrators, instead of adopting a global orchestrator approach. In this way, we argue that the whole orchestration process can experience better results if split among different actors (or orchestrators). 

A key challenge is therefore to develop an orchestration process that is massively scalable. This process could involve one or more orchestrators, becoming open and flexible enough to address future applications and enable the integration with external components. The orchestration must avoid the congestions and bottlenecks in the management and orchestration plane to handle the requests for network services.
 
\subsection{Security and Resiliency}

Softwarized networks modify the way how services are deployed replacing the hardware-based network service components with software-based solutions~\cite{Draxler2017SONATA:Networksb}. Through technologies such as \gls{sdn} and \gls{nfv}, such network can provide automation, programmability, and flexibility. Generally, it depends on centralized control, which leads to risks to security and resiliency~\cite{Arfaoui2017SecurityDirections}. Thus, new protection capabilities need to be put in place, including advanced management capabilities such as authentication, access control, and fault management. 

Security and resiliency must be considered both in design and operation stages of network services. Typically, the services are deployed first, prior to any efforts regarding security development. However, security must be a mandatory issue, mainly in a highly connected and virtualized environment. 

Service instantiation involves automated processes that add and delete network elements and functions without human intervention. A critical problem is the addition of a malicious node that can perform attacks, catch valuable information and even the disruption of the entire services.      

An essential requirement for a multi-domain orchestration platform is the capability to hide specific details of each domain. This ensures privacy and confidentiality of the domains, preserving capabilities and resources to an external component~\cite{francescon2017x}.

Resilience in main NSO components such as orchestrators, controllers, and managers is also a critical problem because it can impact directly in overall service operation. Besides, open interfaces that support network programmability and \gls{nso} components communication with other external elements such as \gls{oss} and other orchestrators are an open issue and a hot topic in research ~\cite{Ordonez-Lucena2017NetworkChallenges}, ~\cite{Arfaoui2017SecurityDirections},~\cite{7345422}. In the same direction, the 5G-PPP published a white paper~\cite{elayoubi:hal-01488208} suggesting that the orchestration platform must be secure, reliable and flexible.
\section{CONCLUSIONS}
\label{sec:Conclusion}

The traditional telecommunication industry is facing multiple challenges to keep competitive and improve the mode network services are designed, deployed and managed. Architectures and enabling technologies such as Cloud Computing, SDN and NFV, are providing new paths to overcome these challenges in a software-driven approach.  \acrfull{nso} is a strategic element to converge various technology domains and provide a broader and more agile network service footprints. 

In this comprehensive survey on network service orchestration, we highlight its growing importance and try to contribute to an overarching understanding of the common concepts and diverse approaches towards practical embodiments of NSO. We present enabling technologies, clarify the definitions behind the term orchestration, review standardization advances, research projects, commercial solutions, and list a number of open issues and research challenges. 

The application of NSO in some scenarios was also presented, where it is possible to sense its potential and understand the motivation behind so much ongoing work. We also observe a growing trend towards the use of open source components or solutions in orchestration platforms; however, the platforms require to evolve until become suitable for production. An important contribution of this work was the definition of a taxonomy that categorizes the leading characteristics and features related to network service orchestration.

Despite the fast pace of this vibrant topic, we expect this survey to serve as a guideline to researchers and practitioners looking into an overview of network service orchestration fundamentals, a reference to relevant related work and pointers to open research questions.

\section*{Acknowledgment}

This research was supported by the Innovation Center, Ericsson S.A., Brazil, grant UNI.62. The authors would also like to express their gratitude to review contributions from David Moura,  Lucian Beraldo, Nazrul Islam, and Suneet Singh (in alphabetical order) funded by the EU-Brazil NECOS project under grant agreement no. 777067.
The authors are thankful for any feedback to improve the work. Do not hesitate to contact the authors and/or via \textit{github}: 
https://github.com/intrig-unicamp/publications/tree/master/NSO-Survey.

\addtolength{\textheight}{-0.1cm}
\Urlmuskip=0mu plus 1mu\relax
\bibliographystyle{IEEEtran}
\bibliography{IEEEabrv,bibtex/bib/refs}

\begin{IEEEbiography}[{\includegraphics[width=1in,height=1.25in,clip,keepaspectratio]{Figures/profile/nathan.jpg}}]{Nathan F. Saraiva de Sousa}
has a BSc and an MSc in Computer Science from Federal University of Piaui (UFPI), in 2005 and 2015, respectively. He is currently a Ph.D. student at the Department of Computer Engineering and Industrial Automation of the School of Electrical and Computer Engineering – UNICAMP. He works in a project funded by Ericsson Research Brazil. Since 2010 he is a Computer Analyst at the Federal Institute of Piaui (IFPI). His research interests span  SDN, NFV, and multi-domain orchestration.
\end{IEEEbiography}

\begin{IEEEbiography}[{\includegraphics[width=1in,height=1.25in,clip,keepaspectratio]{Figures/profile/danny.JPG}}]{Danny Alex Lachos Perez}
is a Ph.D. student in the Faculty of Electrical and Computer Engineering at University of Campinas (UNICAMP). He is involved as a researcher in the 5G Multi-Domain Orchestration project supported by the Innovation Center of Ericsson Telecommunication S.A (Brazil). His current research interests are 5G Networks, Multi-domain Orchestration, SDN, NFV, Machine Learning and Graph Database. He holds a Master degree in Computer Engineering from University of Campinas, Brazil, 2016 and he also received the Computer Engineering degree from Pedro Ruiz Gallo University, Per\'{u}, 2004.
\end{IEEEbiography}

\begin{IEEEbiography}[{\includegraphics[width=1in,height=1.25in,clip,keepaspectratio]{Figures/profile/raphael.png}}]{Raphael V. Rosa}
Currently pursues his thesis on multi-domain distributed NFV as a PhD student in University of Campinas, Brazil. During the last two years, he worked as a visiting researcher in Ericsson Research Hungary, where he contributed to EU-FP7 Unify project and developed activities within H2020 5G Exchange project. His main interests sit on state-of-the-art SDN and NFV research topics. \\
\end{IEEEbiography}

\begin{IEEEbiography}[{\includegraphics[width=1in,height=1.25in,clip,keepaspectratio]{Figures/profile/mateus.JPG}}]{Mateus A. S. Santos} received his Ph.D. from University of Sao Paulo (USP) in 2014. From 2013 to 2014 he was a research scholar with the Inter-Networking Research Group at UC Santa Cruz. He was also a postdoctoral researcher with University of Campinas (UNICAMP) from 2014 to 2016. His research interests are in software-defined networking, network functions virtualization and network security. He is currently with Ericsson Research Brazil.
\end{IEEEbiography}

\begin{IEEEbiography}[{\includegraphics[width=1in,height=1.25in,clip,keepaspectratio]{Figures/profile/chesteve.png}}]{Christian Esteve Rothenberg}
is an Assistant Professor in the Faculty of Electrical \& Computer Engineering (FEEC) at University of Campinas (UNICAMP), Brazil, where he received his Ph.D. and currently leads the Information \& Networking Technologies Research \& Innovation Group (INTRIG). His research activities span all layers of distributed systems and network architectures and are often carried in collaboration with industry, resulting in multiple open source projects in SDN and NFV among other scientific results. Christian has two international patents and over 100  publications, including scientific journals and top-tier networking conferences such as SIGCOMM and INFOCOM.
\\
\end{IEEEbiography}

\end{document}


